\documentclass[11pt]{article}
\usepackage{amsmath}
\usepackage{amsthm}
\usepackage{amsfonts}
\usepackage{amssymb}
\usepackage{graphicx}
%\usepackage{bbold}
\usepackage{url}
\usepackage{geometry}
\def\C{\mathbb{C}}
\def\Z{\mathbb{Z}}
\def\N{\mathbb{N}}
\def\Q{\mathbb{Q}}
\def\R{\mathbb{R}}
\def\Pp{\mathcal{P}}
\def\lcm{\mathrm{lcm}}
\def\proj{\mathrm{proj}}
\def\ord{\mathrm{ord}}
\def\bz{\mathbb{1}}
%\def\1{\mathbb{1}}
\def\Arg{\mathrm{Arg}}
\renewcommand\Re{\mathrm{Re}}
\renewcommand\Im{\mathrm{Im}}
\newcommand{\leg}[2]{\genfrac{(}{)}{}{}{#1}{#2}}
\newcommand{\vc}[1]{\mathbf{#1}}
\theoremstyle{plain}

\newtheorem*{lem}{Lemma}
\newtheorem*{thm}{Theorem}

\theoremstyle{remark}
\newtheorem*{rmk}{Remark}
\geometry{left=1in, right=0.9in, top=.75in, bottom=.75in}
\makeatletter
\let\@@pmod\pmod
\DeclareRobustCommand{\pmod}{\@ifstar\@pmods\@@pmod}
\def\@pmods#1{\mkern4mu({\operator@font mod}\mkern 6mu#1)}
\makeatother
\begin{document}
\thispagestyle{empty} \begin{center} {\textbf{MATH 4000/6000 --
Homework \#4}\\ posted \today; due March 4, 2022}
\end{center}

\begin{quote} {\scriptsize Answer the questions, then question the answers.\\
--- Glenn Stevens}
\end{quote}

\noindent Assignments are expected to be neat and stapled. \textbf{Illegible work may not be marked}. Starred problems (*) are required for those in MATH 6000 and extra credit for those in MATH 4000.


\begin{enumerate}

\item\label{ex:subring} Let $R$ be a ring. A subset $R' \subset R$ is called a \textbf{subring} of $R$ if\\ (A)  $R'$ is a ring for the same operations
$+$ and $\cdot$ as in $R$, \emph{and}\\
(B) $R'$ contains the multiplicative identity $1_R$ of $R$.

(For example, making the identifications via the maps $\phi$ discussed in class, $\Z$ is a subring of $\Q$ and $\R$ is a subring of $\C$.)

\begin{enumerate}
  \item Let $R$ be a ring. Suppose that $R'$ is a subset of $R$ closed under $+$ and $\cdot$, that $R'$ contains the additive inverse of each of its elements, and that $R'$ contains $1_{R}$. Show that $R'$ is a subring of $R$.

{\scriptsize \emph{Hint:} (B) holds by assumption. Check that all the ring axioms hold for $R'$ in order to verify (A). To get started, show that the additive identity of $R$ --- call this $0_R$ --- must belong to $R'$.}

\item Find a two-element subset $R'$ of $R=\Z_6$ that satisfies condition (A) in the definition of a subring but not (B). (You do \textbf{not} have to give a detailed proof that (A) holds.)
\end{enumerate}


\item (Introduction to the Gaussian integers) Let $\Z[i]$ be the subset of complex numbers defined by $\Z[i]:= \{a+bi: a, b \in \Z\}$.
\begin{enumerate}
  \item Check that $\Z[i]$ is a subring of $\mathbb{C}$. (Exercise \ref{ex:subring} above may be helpful.)
  \item Define a function $N\colon \Z[i]\to\R$ by $N(z) = z\cdot \bar{z}$. This is called the \textbf{norm} of $z$. Explain why $N(z)$ is a nonnegative integer for every $z \in \Z[i]$. For which $z\in \Z[i]$ is $N(z)=0$?
 \item Prove that $N(zw) = N(z) N(w)$ for all $z, w \in \Z[i]$.
 \item Using (c), show that $z \in \Z[i]$ is a unit $\Longleftrightarrow$ $N(z)=1$. Then find (with proof) all units in $\Z[i]$.
\end{enumerate}

\item Let $F$ be a field in which $2\ne 0$, and let $a$ be a nonzero element of $F$. Show that the equation $z^2-a=0$ has either no solutions in $F$ or exactly two distinct solutions.
    
\item Recall from class that $(\cos(\theta)+i\sin(\theta))^n = \cos(n\theta) + i\sin(n\theta)$, for every real number $\theta$ and positive integer $n$.

By expanding $(\cos(\theta)+i\sin(\theta))^4$, find formulas for $\cos(4\theta)$ and $\sin(4\theta)$ in terms of $\cos(\theta)$ and $\sin(\theta)$.

\item Let $n \in \Z^{+}$. We say that the complex number $z$ is a \emph{primitive $n$th root of $1$} if
\begin{itemize}
\item[(i)] $z^n=1$, and
\item[(ii)] there is no positive integer $m < n$ with $z^m=1$.
\end{itemize} For example, $-1$ is a primitive $2$nd root of $1$, since $(-1)^2=1$ but $(-1)^1 \ne 1$.

Show that a primitive $n$th root of $1$ exists for every $n$. How many primitive $n$th roots of $1$ are there for $n=1,2,3,4$?

\item 3.1.2(a), and then \\
$f(x) = x^2+2x+2$, $g(x)=x^2+1$, $F=\Z_3$

\item (*) Exercise 2.1.16.

\item (*) Let $k, n$ be positive integers. Show that $\cos(2\pi i k/n) + i \sin (2\pi i k/n)$ is a primitive $n$th root of $1$ if and only if $\gcd(k,n)=1$. 

\end{enumerate}

\end{document}

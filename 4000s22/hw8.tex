\documentclass[11pt]{article}
\usepackage{amsmath}
\usepackage{amsthm}
\usepackage{amsfonts}
\usepackage{amssymb}
\usepackage{graphicx}
\renewcommand\subset\subseteq
%\usepackage{bbold}
\usepackage{url}
\usepackage{geometry}
\def\C{\mathbb{C}}
\def\Z{\mathbb{Z}}
\def\N{\mathbb{N}}
\def\Nm{\mathbf{N}}
\def\Q{\mathbb{Q}}
\def\R{\mathbb{R}}
\def\Pp{\mathcal{P}}
\def\lcm{\mathrm{lcm}}
\def\proj{\mathrm{proj}}
\def\ord{\mathrm{ord}}
\def\bz{\mathbb{1}}
%\def\1{\mathbb{1}}
\def\Arg{\mathrm{Arg}}
\renewcommand\Re{\mathrm{Re}}
\renewcommand\Im{\mathrm{Im}}
\renewcommand\bar\overline
\newcommand{\leg}[2]{\genfrac{(}{)}{}{}{#1}{#2}}
\newcommand{\vc}[1]{\mathbf{#1}}
\theoremstyle{plain}

\newtheorem*{lem}{Lemma}
\newtheorem*{thm}{Theorem}

\theoremstyle{remark}
\newtheorem*{rmk}{Remark}
\geometry{left=1in, right=1in, top=.75in, bottom=.75in}
\makeatletter
\let\@@pmod\pmod
\DeclareRobustCommand{\pmod}{\@ifstar\@pmods\@@pmod}
\def\@pmods#1{\mkern4mu({\operator@font mod}\mkern 6mu#1)}
\makeatother
\begin{document}
\thispagestyle{empty} \begin{center} {\textbf{MATH 4000/6000 --
Homework \#8}\\ posted \today; due April 18, 2022}
\end{center}

\begin{quote} {\scriptsize Many who have never had occasion to learn what mathematics is confuse it with arithmetic, and consider it a dry and arid science. In reality, however, it is the science which demands the utmost imagination.
\\
-- Sofia Kovalevskaya}
\end{quote}

\noindent Assignments are expected to be neat and stapled. \textbf{Illegible work may not be marked}. Starred problems (*) are required for those in MATH 6000 and extra credit for those in MATH 4000.

\vskip 0.1in
\noindent In this assignment, ``ring'' always means ``commutative ring.''
\begin{enumerate}

\item Let $R$ be a ring, and let $I$ be an ideal of $R$. Prove that $R/I$ is the zero ring if and only if $I=R$. (Remember that a ring is called ``the zero ring'' when its additive identity is the same as its multiplicative identity.)

\item In class we stated that isomorphism is an equivalence relation on the class of rings. Here you are asked to show part of this, namely that isomorphism is symmetric.
    
    Specifically, suppose $\phi\colon R \to S$ is an isomorphism. Let $\psi\colon S \to R$ be the inverse function\footnote{Remember that this means $\psi(\phi(r)) = r$ for all $r \in R$ and $\phi(\psi(s)) = s$ for all $s \in S$.} of $R$. Prove that $\psi$ is an isomorphism.


\item
\begin{enumerate}
\item[(a)] Let $R$ be a ring, not the zero ring. We call an ideal $I \subset R$  a \textbf{prime ideal} if
     \begin{enumerate}
     \item[(i)] $I \ne R$,
     \item[(ii)] whenever $a$ and $b$ are elements of $R$ for which $ab \in I$, either $a \in I$ or $b \in I$ (or both).
     \end{enumerate}
Show that for every ideal $I$ of $R$,
\[ \text{$R/I$ is an integral domain $\Longleftrightarrow$ $I$ is a prime ideal of $R$.} \]
\item[(b)] What are all of the prime ideals of $\mathbb{Z}$? Justify your answer.
\end{enumerate}

\item Exercise 4.2.1.

\item Exercise 4.2.6(b).

\item Let $R$ be a ring.
\begin{enumerate}
	\item If $I$ and $J$ are ideals of $R$, we let $I+J = \{i+j: i \in I, j \in J\}$. Show that $I+J$ is an ideal of $R$ and that $I+J$ contains both $I$ and $J$.
	\item Let $a \in R$, and let $I$ be an ideal of $R$. Suppose that $\langle a\rangle + I = R$, where + is addition of ideals as defined in part (a). Show that $\bar{a}$ is a unit in $R/I$.
\end{enumerate}


\item Use the Fundamental Homomorphism Theorem to establish the following ring isomorphisms.
\begin{enumerate}
\item $R/\langle 0\rangle \cong R$ for every ring $R$.
\item $\R[x]/\langle x^2+6\rangle \cong \C$.

{\scriptsize \emph{Hint:} Consider the ``evaluation at $i\sqrt{6}$'' homomorphism taking $f(x) \in \R[x]$ to $f(i\sqrt{6}) \in \C$.}
\item $\Q[x]/\langle x^2-1\rangle \cong \Q \times \Q$.
{\scriptsize \emph{Hint:} Consider the homomorphism from $\Q[x]$ to $\Q\times\Q$ given by $f(x)\mapsto (f(1), f(-1))$.}
\end{enumerate}

\item(*) Let $m$ and $n$ be positive integers. Show that if $\Z_{mn}\cong \Z_m \times \Z_n$, then $\gcd(m,n)=1$. 

\end{enumerate}

\end{document}

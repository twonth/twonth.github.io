\documentclass[11pt]{article}
\usepackage{amsmath}
\usepackage{amsthm}
\usepackage{amsfonts}
\usepackage{amssymb}
\usepackage{graphicx}
\usepackage{cancel}
\renewcommand\subset\subseteq
%\usepackage{bbold}
\usepackage{url}
\usepackage{geometry}
\def\C{\mathbb{C}}
\def\Z{\mathbb{Z}}
\def\N{\mathbb{N}}
\def\Nm{\mathbf{N}}
\def\Q{\mathbb{Q}}
\def\R{\mathbb{R}}
\def\Pp{\mathcal{P}}
\def\lcm{\mathrm{lcm}}
\def\proj{\mathrm{proj}}
\def\ord{\mathrm{ord}}
\def\bz{\mathbb{1}}
%\def\1{\mathbb{1}}
\def\Arg{\mathrm{Arg}}
\renewcommand\Re{\mathrm{Re}}
\renewcommand\Im{\mathrm{Im}}
\newcommand{\leg}[2]{\genfrac{(}{)}{}{}{#1}{#2}}
\newcommand{\vc}[1]{\mathbf{#1}}
\theoremstyle{plain}

\newtheorem*{lem}{Lemma}
\newtheorem*{thm}{Theorem}

\theoremstyle{remark}
\newtheorem*{rmk}{Remark}
\geometry{left=1in, right=1in, top=.75in, bottom=.75in}
\makeatletter
\let\@@pmod\pmod
\DeclareRobustCommand{\pmod}{\@ifstar\@pmods\@@pmod}
\def\@pmods#1{\mkern4mu({\operator@font mod}\mkern 6mu#1)}
\makeatother
\begin{document}
\thispagestyle{empty} \begin{center} {\textbf{MATH 4000/6000 --
Homework \#6}\\ posted \today; due March 30, 2022}
\end{center}

\begin{quote} {\scriptsize The beauty of mathematics only shows itself to more patient followers. -- Maryam Mirzakhani}
\end{quote}

\noindent Assignments are expected to be neat and stapled. \textbf{Illegible work may not be marked}. Starred problems (*) are required for those in MATH 6000 and extra credit for those in MATH 4000.

\vskip 0.1in
\begin{enumerate}

\item Exercise 3.2.1.

\item Exercise 3.2.6(a,e).

\item 
\begin{enumerate}
\item Show that $\Q[\sqrt{2},i]$ is a splitting field for $x^8-1$ over $\Q$. 

{\scriptsize \emph{Hint}. We know from class that $\Q[\cos(2\pi/8) + i\sin(2\pi/8)]$ is a splitting field for $x^8-1$ over $\Q$. What are $\cos(2\pi/8)$ and $\sin(2\pi/8)$?}
\item Let $\zeta = \cos(2\pi/2022) + i \sin(2\pi/2022)$, and let $\sqrt[2022]{3}$ denote the positive real 2022th root of $3$. Prove that $\Q[\zeta,\sqrt[2022]{3}]$ is a splitting field for $x^{2022}-3$ over $\Q$.
\end{enumerate}

\item Exercise 3.3.2(b,c,e,h).

\item Exercise 3.3.4.

\item (*) Exercise 3.3.7.

{\scriptsize \emph{Hint}. Argue that the Eisenstein criterion can be applied to $f(x+1)$. Look at Examples 7(c) on p. 110.}


\end{enumerate}

\end{document}

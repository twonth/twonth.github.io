\documentclass[11pt]{article}
\usepackage{amsmath}
\usepackage{amsthm}
\usepackage{amsfonts}
\usepackage{amssymb}
\usepackage{graphicx}
%\usepackage{bbold}
\usepackage{url}
\usepackage{geometry}
\def\C{\mathbb{C}}
\def\Z{\mathbb{Z}}
\def\N{\mathbb{N}}
\def\Nm{\mathbf{N}}
\def\Q{\mathbb{Q}}
\def\R{\mathbb{R}}
\def\Pp{\mathcal{P}}
\def\lcm{\mathrm{lcm}}
\def\proj{\mathrm{proj}}
\def\ord{\mathrm{ord}}
\def\bz{\mathbb{1}}
%\def\1{\mathbb{1}}
\def\Arg{\mathrm{Arg}}
\renewcommand\Re{\mathrm{Re}}
\renewcommand\Im{\mathrm{Im}}
\newcommand{\leg}[2]{\genfrac{(}{)}{}{}{#1}{#2}}
\newcommand{\vc}[1]{\mathbf{#1}}
\theoremstyle{plain}

\newtheorem*{lem}{Lemma}
\newtheorem*{thm}{Theorem}

\theoremstyle{remark}
\newtheorem*{rmk}{Remark}
\geometry{left=1in, right=1in, top=.75in, bottom=.75in}
\makeatletter
\let\@@pmod\pmod
\DeclareRobustCommand{\pmod}{\@ifstar\@pmods\@@pmod}
\def\@pmods#1{\mkern4mu({\operator@font mod}\mkern 6mu#1)}
\makeatother
\begin{document}
\thispagestyle{empty} \begin{center} {\textbf{MATH 4000/6000 --
Homework \#7}\\ posted \today; due April 11, 2022}
\end{center}

\begin{quote}{\scriptsize You can observe a lot by just looking. -- Yogi Berra }
\end{quote}

\noindent Assignments are expected to be neat and stapled. \textbf{Illegible work may not be marked}. Starred problems (*) are required for those in MATH 6000 and extra credit for those in MATH 4000.

\vskip 0.1in
\noindent In this assignment, ``ring'' always means ``commutative ring.''
\begin{enumerate}

\item\label{ex:FGideal} Let $R$ be a ring. Recall that if $x_1,\dots, x_n$ are elements of $R$, then (by definition)
\[ \langle x_1, \dots, x_n \rangle = \{r_1 x_1 +\dots + r_n x_n: \text{all $r_i \in R$}\}. \]
In other words, $\langle x_1,\dots,x_n\rangle$ is the set of all linear combinations of $x_1,\dots,x_n$ with coefficients from $R$. Prove that $\langle x_1,\dots,x_n\rangle$ is an ideal of $R$ by directly verifying the three defining properties of an ideal.

\item Exercise 4.1.3. (In part (c), assume $R$ is not the zero ring.)

\item Suppose $R$ be a ring in which every ideal is principal. That is, every ideal of $R$ has the form $\langle r\rangle$ for some $r \in R$.

Let $x_1, \dots, x_n \in R$. Since $\langle x_1,\dots,x_n\rangle$ is an ideal of $R$, there is some $d \in R$ with $\langle x_1,\dots,x_n\rangle = \langle d\rangle$. Prove that $d$ divides all of $x_1,\dots,x_n$ and that if $d'$ is any element of $R$ dividing all of $x_1,\dots,x_n$, then $d' \mid d$.


\item Let $F$ be a field. Prove that if $I$ is any ideal of $F[x]$, then $I = \langle f(x)\rangle$ for some $f(x) \in F[x]$. 



\item
\begin{enumerate}
\item Let $R$ be an integral domain. Show that if $a, b \in R$, then $\langle a\rangle = \langle b\rangle$ if and only if $a=u\cdot b$ for some unit $u \in R$. 
    
{\scriptsize \emph{Hint}. First show that $\langle a\rangle = \langle b\rangle$ if and only if $a\mid b$ and $b\mid a$.}
\item Now let $R=F[x]$. Show that $\langle a(x)\rangle = \langle b(x)\rangle$, where $a(x), b(x) \in F[x]$, if and only if $a(x) = c\cdot b(x)$ for some nonzero $c \in F$.
\end{enumerate}

\item Let $F$ be a field and suppose that $f(x)\in F[x]$ has degree $n\ge 1$. In class, we showed that the elements of $F[x]/\langle f(x)\rangle$ all have the form $\overline{a_0 + a_1 x + \dots + a_{n-1} x^{n-1}}$, where $a_0, \dots, a_{n-1} \in F$. Show that this representation is unique; that is, distinct choices of $a_i$ lead to distinct elements of $F[x]/\langle f(x)\rangle$.


\item Exercise 4.1.14(c). Make sure to answer the two questions at the end (is it a field? is it an integral domain?).

\item Exercise 4.1.10. 

{\scriptsize \emph{Hint.} If you get stuck, try Exercise 4.1.9 first.}

\item Let $F$ be a field, and let $f(x) \in F[x]$ be irreducible. Show that $F[x]/\langle f(x)\rangle$ is a field. 

{\scriptsize {\emph Hint}. If $f(x) \nmid a(x)$, then there are $X(x), Y(x) \in F[x]$ with $a(x) X(x) + f(x) Y(x)=1$. What does this equation tell you in $F[x]/\langle f(x)\rangle$?}

\item (*) Let $R = \Z[x]$, and let $I$ be the set of elements of $R$ with even constant term. Show that $I$ is an ideal of $R$ but that $I$ is not principal: there is no $f(x) \in \Z[x]$ with $I = \langle f(x) \rangle$.

\end{enumerate}



\end{document}

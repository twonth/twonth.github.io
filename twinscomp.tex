\documentclass[a4paper]{compositio}
\usepackage{amsmath,amssymb,amsthm,amscd}
\usepackage[curve]{xy}
\begin{document}
\renewcommand{\theenumi}{\roman{enumi}}
\newtheorem*{abh}{Abhyankar's Lemma}
\newtheorem*{hl}{A Hardy-Littlewood Conjecture for Polynomials over $\F_q$}
\newtheorem*{bang}{Bang's Theorem on Primitive Prime Divisors}
\newtheorem*{bsd}{A Criterion of Birch \& Swinnerton-Dyer}
\newtheorem*{cast}{Castelnuovo's Inequality}
\newtheorem*{cheb}{Explicit Chebotarev Density Theorem for First Degree Primes}
\newtheorem*{hyph}{Hypothesis H}
\newtheorem*{thma}{Theorem A}
\newtheorem*{guess}{Conjecture}
\newtheorem{thm}{Theorem}
\newtheorem*{theorem}{Theorem}
\newtheorem*{corollary}{Corollary}
\newtheorem{conj}[thm]{Conjecture}
\newtheorem{lem}[thm]{Lemma}
\newtheorem{cor}[thm]{Corollary}
\newtheorem{prop}[thm]{Proposition}
\theoremstyle{remark}
\newtheorem*{example}{Example}
\theoremstyle{definition}
\newtheorem*{rmk}{Remark}
\newtheorem*{rmks}{Remarks}
\def\sing{\mathfrak{S}}
\def\disc{\mathrm{disc}}
\def\res{\mathrm{res}}
\def\Frob{\mathrm{Frob}}
\def\Gal{\mathrm{Gal}}
\def\Sym{\mathrm{Sym}}
\def\I{\mathcal{I}}
\def\P{\mathcal{P}}
\def\Z{\mathbf{Z}}
\def\R{\mathbf{R}}
\def\Q{\mathbf{Q}}
\def\F{\mathbf{F}}
\def\C{\mathcal{C}}
\def\Nm{\mathrm{Nm}}
\def\Tr{\mathrm{Tr}}
\def\lcm{\mathrm{lcm}}
\newcommand{\leg}[2]{\genfrac{(}{)}{}{}{#1}{#2}}
\title[Counting Irreducibility-preserving substitutions]{Counting irreducibility-preserving substitutions for polynomials over
finite fields}
\author{Paul Pollack}
\email{paul.pollack@dartmouth.edu} \thanks{The author is supported
by an NSF Graduate Research Fellowship.}
\begin{abstract} Let $n$ be a positive integer and let $f_1(T), \dots, f_r(T)$ be
pairwise nonassociated irreducible polynomials over a finite field
$\F_q$ with the degree of $f_1 \cdots f_r$ bounded by $B$.  We show
that the number of univariate monic polynomials $h$ of degree $n$
for which all of $f_1(h(T)), \dots, f_r(h(T))$ are irreducible over
$\F_q$ is $q^n/n^r + O_{n,B}(q^{n-1/2})$ provided $\gcd(q,2n)=1$. As
an application, fix an infinite arithmetic progression $a \bmod{m}$
and fix pairwise nonassociated irreducibles $f_1(T), \dots, f_r(T)$
over $\F_p$ with the degree of $f_1\cdots f_r$ bounded by $B$. If
$p$ is sufficiently large depending only on $m$, $r$, and $B$, then
there are infinitely many monic polynomials $h(T)$ with
$\deg{h}\equiv a \pmod{m}$ and all of $f_1(h(T)), \dots, f_r(h(T))$
irreducible over $\F_p$.
\end{abstract}
\keywords{Finite fields, Hardy-Littlewood conjectures, Irreducible
polynomials, Twin prime polynomials}
\address{Department of Mathematics\\Dartmouth College\\Hanover, NH 03755}
\subjclass[2000]{Primary: 11T55, Secondary: 11N32}\maketitle
\section{Introduction}

Are there infinitely many primes of the form $n^2+1$? Questions of
this type, where one inquires about the prime values of a polynomial
(or the simultaneous prime values of a finite collection of
polynomials) have received considerable attention, especially since
the development of sieve methods in the early 20th century. Yet we
still cannot prove the existence a single polynomial of degree $> 1$
that assumes prime values infinitely often.

In 1923, Hardy \& Littlewood \cite{hl23} formulated quantitative
predictions for the number of simultaneous prime values assumed on
integers $n\leq x$ for several specific families of polynomials. A
general prediction for all finite collections of polynomials was
later given by Bateman \& Horn \cite{bh}; roughly speaking, the
number of such $n$ is conjectured to be governed by a global factor
predicted by the density of primes, multiplied by a local factor
depending on the number of solutions of our polynomials modulo $p$
for all primes $p$.

In light of the strong analogies between the ring of integers and
the ring of polynomials in one variable over a finite field, it is
natural to wonder if similar conjectures can be formulated in the
polynomial context. This is indeed the case: the following is one
plausible analogue of the Hardy-Littlewood/Bateman-Horn conjectures:

\begin{conj}[(A Hardy-Littlewood Conjecture for Polynomials over $\F_q$)]\label{conj:hl} Let $f_1, \dots, f_r$ be nonassociated irreducible
one-variable polynomials over $\F_q$. Suppose that there is no prime
$\pi$ of $\F_q[T]$ for which the map
\[ h(T) \mapsto f_1(h(T)) \cdots f_r(h(T)) \bmod \pi \]
is identically zero. Then there are infinitely many monic $h(T)$ for
which all of $f_1(h(T)), \dots, f_r(h(T))$ are simultaneously
irreducible over $\F_q$. Moreover,
\begin{multline*} \#\{h(T): h \text{ monic},~\deg{h}=n, \text{ and $f_1(h(T)), \dots, f_r(h(T))$
are all prime}\} \sim
\\ \sing(f_1,\dots,f_r)  \frac{1}{\prod_{i=1}^{r}\deg{f_i}}\frac{q^n}{n^r}  \quad
\text{as $n\to\infty$}. \end{multline*} Here the local factor
$\sing(f_1, \dots, f_r)$ is defined by
\[ \sing(f_1, \dots, f_r) := \prod_{n=1}^{\infty}\prod_{\substack{\deg{\pi}=n \\ \pi \text{ monic prime of
$\F_q[T]$}}}\frac{1- \omega(\pi)/q^{n}}{(1-1/q^{n})^r},
\] where
\[ \omega(\pi) := \#\{a \bmod{\pi}: f_1(a) \cdots f_r(a) \equiv 0
\pmod{\pi}\}. \]
\end{conj}
\begin{rmks} We make two remarks about the behavior of $\sing(f_1, \dots, f_r)$, deferring their proofs
to \S\ref{sec:local}.
\begin{enumerate}
\item Under the hypotheses of Conjecture \ref{conj:hl}, the product defining
$\sing(f_1,\dots,f_r)$ converges to a nonzero constant.
\item If the sum of the degrees of the $f_i$ is bounded, then the ratio $\sing(f_1, \dots,
f_r)/\prod_{i=1}^r{\deg{f_i}}$ tends uniformly to $1$ as $q$ tends
to infinity. This observation will be useful in explaining the form
of Theorem \ref{thm:main} below.
\end{enumerate}
\end{rmks}

Conjecture \ref{conj:hl} is the expected translation of the
Hardy-Littlewood/Bateman-Horn prediction into the polynomial
setting, except that we have been a bit conservative in our
formulation by restricting ourselves to polynomials with
coefficients from $\F_q$. A complete analogue of the
Hardy-Littlewood conjectures would address prime specializations of
polynomials with coefficients from $\F_q[u]$ (predicting, e.g., the
frequency of polynomials $h$ for which $h(u)$ and $h(u)^q+u$ are
both prime in $\F_q[u]$). However, formulating a plausible
conjecture in complete generality requires some care; simply
translating the classical conjectures into the language of
polynomials is no longer adequate. Indeed, the number of prime
specializations of a single irreducible polynomial in $\F_q[u][T]$
may already display unexpected behavior if the polynomial is
inseparable over $\F_q(u)$. The underlying issues here were brought
to light and vigorously explored by Conrad, Conrad \& Gross
(\cite{ccg03}; see also the survey \cite{conrad05}). We restrict our
attention in this paper to Conjecture \ref{conj:hl}.

Let $B$ denote an upper bound on the degree of the product
$f_1\cdots f_r$. Then Conjecture \ref{conj:hl} provides a
(predicted) asymptotic formula for fixed $q$ and $B$ valid as
$n\to\infty$. It makes equally good sense to ask for asymptotics in
other ranges of $(q,n,B)$-space, perhaps for a uniform conjecture.
In this paper we make some progress in this direction by proving
asymptotic results when $q$ is large compared to $n$ and $B$,
subject to mild restrictions on the characteristic of $\F_q$. Our
main result is as follows:

\begin{thm}\label{thm:main} Let $n$ be a positive integer. Let $f_1(T), \dots, f_r(T)$ be
pairwise nonassociated irreducible polynomials over $\F_q$ with the
degree of the product $f_1 \cdots f_r$ bounded by $B$. The number of
univariate monic polynomials $h$ of degree $n$ for which all of
$f_1(h(T)), \dots, f_r(h(T))$ are irreducible over $\F_q$ is
$q^n/n^r + O_{n,B}(q^{n-1/2})$ provided $\gcd(q,2n)=1$.
\end{thm}
\noindent The dependence of the $O_{n,B}$-term here is explicit but
unpleasant, and it would be interesting to improve this. Observe
that there is no coefficient appearing in front of $q^n/n^r$ in the
estimate of Theorem \ref{thm:main}. Referring to Remark (ii) above,
we see that this is exactly what we would expect if a uniform
version of Conjecture \ref{conj:hl} holds.

\begin{example} The polynomial $T^2+1$ is irreducible over
$\F_q$ if and only if $q \equiv 3 \pmod{4}$. By Theorem
\ref{thm:main} (with $r=1$) the number of $h$ of a fixed degree
$n\geq 1$ for which $h^2+1$ is irreducible is asymptotic to $q^n/n$,
provided $q\to\infty$ through prime powers $3\pmod{4}$ satisfying
$\gcd(q,2n)=1$. This prediction may be initially surprising: if $h$
has degree $n$, then $h^2+1$ has degree $2n$, and a random
polynomial of degree $2n$ is irreducible with probability roughly
$1/(2n)$. So we obtain from Theorem \ref{thm:main} twice as many
irreducible specializations as we might expect. But this naive
expectation fails to take the local data into account; as remarked
above, a uniform version of Conjecture \ref{conj:hl} would explain
the discrepancy. Alternatively, one can convince oneself that the
estimate of Theorem \ref{thm:main} is reasonable by realizing that
for $q\equiv 3 \pmod{4}$, the polynomial $h^2+1$ is irreducible over
$\F_q$ exactly when $h+i$ is irreducible over $\F_{q^2}$, and it is
plausible that the latter should happen with probability about
$1/n$.
\end{example}

Theorem \ref{thm:main} was inspired by the result of Effinger,
Hicks, and Mullen \cite{ehm02} that for each fixed $n \geq 1$ and
every large enough finite field $\F_q$, one can find a pair of
distinct monic irreducibles of degree $n$ over $\F_q$ which differ
only in their constant term. To see this, let $h(T)$ range over the
polynomials of degree $n$ with vanishing constant term, and let
$N_h$ denote the number of $a \in \F_q$ for which $h(T)-a$ is
irreducible over $\F_q$. By Gauss's formula for the number of monic
irreducibles of degree $n$ (see, for example, \cite[Theorem
3.25]{ln97}), we have
\[ \sum_{h} N_{h} = q^n/n + O(q^{n/2}/n), \]
so that by the Cauchy-Schwarz inequality,
\[ \sum_{h}{1^2} \sum_{h} N_{h}^2 \geq \left(\sum_{h} N_{h}\right)^2=
q^{2n}/n^2 + O(q^{3n/2}/n^2), \] and hence
\[ \sum_{h} N_{h}^2 \geq q^{n+1}/n^2 + O(q^{n/2+1}/n^2). \]
But the left-hand side counts the number of ordered pairs of monic
degree $n$ irreducibles which differ at most in their constant term.
Since the lower bound exceeds the number of trivial pairs once $q$
is large enough compared to $n$, the result follows.

An averaging argument of this kind does not appear sufficient for
the proof of Theorem \ref{thm:main}. Instead we employ an explicit
form of the Chebotarev density theorem. Our argument is similar in
strategy to that used by Cohen \cite{cohen70} and Ree (\cite{ree71},
\cite{ree72}) to settle Chowla's conjecture \cite{chowla66} on the
existence of prime polynomials of the form $T^n + T + a$ modulo $p$
for $p > p_0(n)$.

Under the hypotheses of Conjecture \ref{conj:hl} we expect
infinitely many irreducibility-preserving specializations.
Surprisingly, this qualitative version of Conjecture \ref{conj:hl}
can be rigorously confirmed in many special  cases, even if the
asymptotic appears out of reach. The first to make significant
progress in this direction was Hall \cite{hall06}, who showed that
there are infinitely many monic twin prime pairs $f, f+1$ over all
finite fields $\F_q$ with more than two elements (excepting $\F_3$,
which was later treated by the present author \cite[Theorem
1]{pollack06a}). Generalizing the work of Hall, the author recently
established the following result (cf. \cite[Theorem 2]{pollack06a}):
\begin{thma} Let $f_1(T), \dots, f_r(T)$ be
pairwise nonassociated irreducible polynomials over $\F_q$ with the
degree of $f_1 \cdots f_r$ bounded by $B$. If $q \geq 2^{2r} \left(1
+ B\right)^2$, then there is a prime $l$ dividing $q-1$ and an
element $\beta \in \F_q$ for which every substitution
\[ T \mapsto T^{l^k}-\beta \quad \text{with} \quad k=1, 2, 3, \dots \]
leaves all of $f_1, \dots, f_r$ irreducible. In particular, there
are infinitely many $h$ as in Conjecture \ref{conj:hl}.
\end{thma}
In both Hall's original theorem and in Theorem A, the set of
substitutions $T \mapsto h(T)$ leaving all the $f_i$ irreducible
form a sparse set. A weak consequence of Conjecture \ref{conj:hl} is
that there should be such irreducibility-preserving substitutions
$h(T)$ of every sufficiently large degree. Here we establish that
the degrees of the permissible substitutions are ``dense'' with
respect to arithmetic progressions, in the following sense:

\begin{thm}\label{thm:arith} Let $f_1(T), \dots, f_r(T)$ be pairwise
nonassociated irreducibles over $\F_q$ with the degree of $f_1
\cdots f_r$ bounded by $B$. Let $a \bmod m$ be an arbitrary infinite
arithmetic progression of integers. If the finite field $\F_q$ is
sufficiently large, depending just on $m$, $r$, and $B$, and if $q$
is prime to $2\gcd(a,m)$, then there are infinitely many univariate
monic polynomials $h$ over $\F_q$ with
\[ \deg{h} \equiv a \pmod{m} \quad \text{and} \quad f_1(h(T)), \dots, f_r(h(T)) \text{ all irreducible over
$\F_q$}. \]
\end{thm}
This result is most satisfactory in the case of prime fields, since
the restriction that $q$ be coprime to $2\gcd(a,m)$ is satisfied for
all sufficiently large primes $q$. Probably Theorem \ref{thm:arith}
remains true without any restriction on the characteristic of
$\F_q$, but we have not been able to show this.

Our proof of Theorem \ref{thm:arith} is completely effective. We
illustrate our methods with the following result, the first half of
which settles a problem posed by Hall \cite[p. 140]{hall06}:
\begin{thm}\label{thm:hall} Let $\F_q$ be any finite field with more than two elements.
Then there are infinitely many monic prime pairs $f, f+1$ of odd
degree over $\F_q$. The same holds for the case of even degree.
\end{thm}
\noindent Even for large $q$ this is not immediate from Theorem
\ref{thm:arith}, since that theorem says nothing about prime
specializations over fields of characteristic $2$.

Theorem \ref{thm:hall} is an analogue of Kornblum's result that
every coprime residue class of polynomials over $\F_q$ contains
infinitely many monic irreducibles of odd degree, as well as
infinitely many of even degree. In the posthumously-published
version of Kornblum's paper \cite{kornblum}, Landau proved the more
general theorem that the degrees can be taken from an arbitrary
arithmetic progression. Theorem \ref{thm:arith} can be seen as an
effort in the same direction.

\section{Analysis of the Local Factor}\label{sec:local}
Here we justify the remarks following Conjecture \ref{conj:hl}.
Proposition \ref{prop:prod1} vindicates our first remark, and
Proposition \ref{prop:prod2} establishes the second in a more
precise form.
\begin{prop}\label{prop:prod1} Let $f_1, \dots, f_r$ be pairwise nonassociated
irreducibles over $\F_q$ satisfying also the hypothesis of
Conjecture \ref{conj:hl} that $f_1\cdots f_r$ has no fixed prime
divisor. Then the product defining $\sing(f_1, \dots, f_r)$
converges to a positive constant.
\end{prop}

\begin{prop}\label{prop:prod2} Let $f_1, \dots, f_r$ be pairwise nonassociated
irreducibles over $\F_q$ with the degree of $f_1\dots f_r$ bounded
by $B$. If $q\geq 2B^2$, then
\[ \frac{1}{\prod_{i=1}^{r}\deg{f_i}} \sing(f_1, \dots, f_r) = 1 + O(B/q^{1/2}), \]
where the implied constant is absolute.
\end{prop}

The proofs of both propositions are based on the following technical
lemma:

\begin{lem}\label{lem:estimate} Let $f_1, \dots, f_r$ be pairwise nonassociated
irreducibles over $\F_q$. For $\pi$ a prime of $\F_q[T]$, let
$\omega(\pi)$ denote the number of roots of $f_1\cdots f_r$ modulo
$\pi$, and for each $i=1,2, \dots, r$, let $\omega_i(\pi)$ denote
the number of roots of $f_i$ modulo $\pi$. Suppose that $B$ is an
upper bound for the degree of $f_1\cdots f_r$. Then
\begin{equation}\label{eq:sprod} \log \prod_{n=N_0}^{N_1} \prod_{\deg{\pi}=n}
\left(1 - \frac{1}{q^n}\right)^{-r} = r
\sum_{n=N_0}^{N_1}\frac{1}{n} + O\left(\frac{B}{N_0
q^{N_0/2}}\right)
\end{equation} for all positive integers $N_0 < N_1$. Moreover, if
$q^{N_0} \geq 2B$, then
\begin{multline}\label{eq:sprod2}
 \log \prod_{n=N_0}^{N_1} \prod_{\deg{\pi}=n} \left(1 -
\frac{\omega(\pi)}{q^n}\right) = \\
-\left(\deg{f_1}\sum_{\substack{n=N_0
\\ \deg{f_1} \mid n}}^{N_1}\frac{1}{n} +\dots+ \deg{f_r}\sum_{\substack{n=N_0
\\ \deg{f_r} \mid n}}^{N_1}\frac{1}{n} \right) +
O\left(\frac{B}{N_0 q^{N_0/2}}\right) + O\left(\frac{B^2}{N_0
q^{N_0}}\right).
\end{multline} All implied constants here
are absolute.
\end{lem}
\begin{proof} The left hand side of \eqref{eq:sprod} is given by
\[ - r \sum_{n=N_0}^{N_1}\sum_{\deg{\pi}=n} \log(1-1/q^n)=  -r \sum_{n=N_0}^{N_1} \sum_{\deg{\pi}=n} \left(-\frac{1}{q^n} + O\left(\frac{1}{q^{2n}}\right)\right).
\] Inserting the estimate $q^n/n + O(q^{n/2}/n)$ for the number of monic irreducibles of
degree $n$, this simplifies to
\[ r \sum_{n=N_0}^{N_1} \frac{1}{n} + O\left(r \sum_{n=N_0}^{N_1} \frac{1}{n q^n} + r \sum_{n=N_0}^{N_1} \frac{1}{n q^{n/2}}\right). \]
The error term is $O(r/(N_0 q^{N_0/2}))$; since $r$ is bounded by
$B$, this proves \eqref{eq:sprod}.

The proof of \eqref{eq:sprod2} is similar but a bit more involved.
Since $q^{N_0} \geq 2B$, we have $\omega(\pi)/q^n \leq 1/2$ for any
prime $\pi$ of $\F_q[T]$ of degree $n \geq N_0$. As a consequence,
we may write the left hand side of \eqref{eq:sprod2} as
\[ -\sum_{n=N_0}^{N_1} \sum_{\deg\pi = n}\left(\frac{\omega(\pi)}{q^n} + O\left(\frac{B^2}{q^{2n}}\right)\right). \]
The $O$-term here is
\begin{equation}\label{eq:oterm}
 \ll \sum_{n=N_0}^{N_1} \frac{B^2}{q^{2n}} \frac{q^n}{n} \ll B^2 \frac{1}{N_0 q^{N_0}}.
\end{equation}
To evaluate the main term, we observe that since the $f_i$ are
pairwise coprime, we have $\omega(\pi) = \omega_1(\pi) + \dots +
\omega_r(\pi)$. Moreover, $\omega_i(\pi)$ vanishes unless the degree
of $f_i$, say $d_i$, divides the degree of $\pi$, in which case
$\omega_i(\pi)=d_i$. With this information in hand, we can write the
main term as
\[ -\sum_{n=N_0}^{N_1} \sum_{\deg\pi=n} \frac{\omega(\pi)}{q^n} = -
\left(d_1 \sum_{\substack{n=N_0 \\ d_1 \mid
n}}^{N_1}\frac{1}{q^n}\sum_{\deg{\pi}=n}{1} + \dots +d_r
\sum_{\substack{n=N_0 \\ d_r \mid
n}}^{N_1}\frac{1}{q^n}\sum_{\deg{\pi}=n}{1}\right).\] For each $1
\leq i \leq r$, we have
\begin{align*}
 d_i \sum_{\substack{n=N_0 \\ d_i \mid
n}}^{N_1}\frac{1}{q^n}\sum_{\deg{\pi}=n}{1} &= d_i
\sum_{\substack{n=N_0 \\ d_i \mid n}}^{N_1}\frac{1}{n} + O\left(d_i
\sum_{n=N_0}^{N_1}\frac{1}{n q^{n/2}} \right) \\ &= d_i
\sum_{\substack{n=N_0 \\ d_i \mid n}}^{N_1}\frac{1}{n} +
O\left(\frac{d_i}{N_0 q^{N_0/2}}\right).
\end{align*}
Adding all these estimates (and keeping in mind the prior error term
\eqref{eq:oterm}) we obtain the main term of \eqref{eq:sprod2} with
an error that is $\ll B/(N_0 q^{N_0/2}) + B^2/(N_0 q^{N_0})$, as
claimed.
\end{proof}

\begin{proof}[Proof of Proposition \ref{prop:prod1}] Suppose that
$q$ and the $f_i$ are fixed, and let $d_i$ denote the degree of
$f_i$ as above. By \eqref{eq:sprod} and \eqref{eq:sprod2}, if $N_1
> N_0$ and $N_0$ is large enough to satisfy $q^{N_0} \geq 2B$, then
we have
\begin{multline}\label{eq:part} \log \prod_{n=N_0}^{N_1}
\prod_{\deg\pi=n}\frac{\left(1-\omega(\pi)/q^n\right)}{(1-1/q^n)^r}
=\\ -\left(d_1 \left(\frac{1}{d_1} \sum_{N_0/d_1 \leq n \leq
N_1/d_1}\frac{1}{n}\right) +\dots + d_r \left(\frac{1}{d_r}
\sum_{N_0/d_r \leq n \leq N_1/d_r}\frac{1}{n}\right)\right) +\\ r
\sum_{n=N_0}^{N_1} \frac{1}{n} + O\left(\frac{B}{N_0 q^{N_0/2}} +
\frac{B^2}{N_0 q^{N_0}}\right).
\end{multline}
Since we have $\sum_{N_0/d_i \leq n \leq N_1/d_i}\frac{1}{n} =
\log{(N_1/N_0)} + O(d_i/N_0)$ while $\sum_{N_0 \leq n \leq
N_1}\frac{1}{n} = \log{(N_1/N_0)} + O(1/N_0)$, this simplifies to
\[ -r \log{(N_1/N_0)} + r \log{(N_1/N_0)} + O\left(B/N_0 + r/N_0 +
\frac{B}{N_0 q^{N_0/2}} + \frac{B^2}{N_0 q^{N_0}}\right) =
O(B^2/N_0).\] But in our situation $B$ is fixed. It follows that
\eqref{eq:part} tends to zero as $N_0$ tends to infinity. This
verifies that the sequence of logarithms of the partial products for
the expression defining $\sing(f_1, \dots,f_r)$ is Cauchy, hence
convergent. Consequently, the product defining $\sing(f_1, \dots,
f_r)$ converges to a positive real number, as claimed. (Note that in
order to know that the sequence of logarithms of the partial
products is well-defined, we are implicitly using the condition that
$f_1\cdots f_r$ has no fixed prime divisor.)
\end{proof}

\begin{proof}[Proof of Proposition \ref{prop:prod2}] We appeal to
estimates \eqref{eq:sprod} and \eqref{eq:sprod2} with $N_0 = 1$.
Since $q\geq 2B^2$, the condition $q^{N_0} \geq 2B$ of Lemma
\ref{lem:estimate} is certainly satisfied. Proceeding as in the
proof of Proposition \ref{prop:prod1}, we find \[ \log
\prod_{n=1}^{N}\prod_{\deg{\pi}=n}\frac{\left(1-\omega(\pi)/q^n\right)}{(1-1/q^n)^r}
= - \left(\sum_{n \leq N/d_1}\frac{1}{n} + \dots + \sum_{n  \leq
N/d_r}\frac{1}{n}\right) + r \sum_{n \leq N} \frac{1}{n} +
O\left(\frac{B}{q^{1/2}} + \frac{B^2}{q}\right). \]Since $\sum_{n
\leq N/d_i} \frac{1}{n} = \log{(N/d_1)} + \gamma + O(d_i/N)$ while
$\sum_{n \leq N}\frac{1}{n} = \log{N} + \gamma + O(1/N)$, the last
expression can be estimated as
\begin{multline*}
 -\log{\frac{N^r}{d_1 \cdots d_r}} + r \log{N} -r\gamma + r\gamma + O(B/N + r/N + B/q^{1/2} + B^2/q) =\\ \log{d_1\cdots
 d_r} + O(B/N + B/q^{1/2}).
\end{multline*}
(Note that $B/q^{1/2} \geq B^2/q$ since $q \geq B^2$.) Letting $N$
tend to infinity and exponentiating now gives the result.
\end{proof}

\section{Preparation for the Proof of Theorem \ref{thm:main}}
\subsection{Notation} We fix once and for all an algebraically closed
field $\Omega_q$ of infinite transcendence degree over $\F_q$ and
assume for the remainder of the paper that all extensions of $\F_q$
which appear are subfields of $\Omega_q$. We use an overline to
denote the operation of taking an algebraic closure; in particular,
$\overline{\F}_q$ denotes the algebraic closure of $\F_{q}$ inside
$\Omega_q$.

We use $\res$ and $\disc$ to denote the polynomial resultant and
discriminant, respectively. Our work also requires variants of these
quantities, which we define as follows: If $f = \sum_{i=0}^{n}a_i
u^i$ and $g = \sum_{j=0}^{m}{b_j u^j}$ are polynomials in $u$ of
degrees \emph{at most} $n$ and $m$ respectively over a domain $R$
(so that $a_n$ and $b_m$ may vanish), we define
\[ \res_u^{n,m}(f,g):= \left.\res_u\left(\sum_{i=0}^{n}{A_i u^i},
\sum_{j=0}^{m} B_j u^j\right)\right|_{A_0= a_0, \dots, A_n= a_n,
B_0=b_0, \dots, B_m = b_m},
\] where the right-hand resultant is computed
over the ring $R[A_0, \dots, A_n, B_0, \dots, B_m]$ of polynomials
obtained by adjoining the indeterminates $A_i$ and $B_j$ to $R$.
Similarly, if $f = \sum_{i=0}^{n}a_i T^i$ is a polynomial in $T$ of
degree at most $n$, we define
\[ \disc_T^n(f) := \left.\disc_T\left(\sum_{i=0}^{n}{A_i T^i}\right)\right|_{A_0 = a_0, \dots,
A_n = a_n}, \] the right-hand discriminant being taken over
$R[A_0,\dots,A_n]$. If $n$ and $m$ represent the actual degrees of
$f$ and $g$, respectively, then $\res_u^{n,m}(f,g) = \res_{u}(f,g)$,
and similarly for $\disc_T^{n}(f)$. We work with $\res_u^{n,m}$ and
$\disc_T^n$ rather than the usual resultant and discriminant in
order to obtain uniform formulas without needing to worry about
``degree-dropping'' in intermediate calculations. The fundamental
property of $\res_u^{n,m}$ that we need is that $\res_u^{n,m}(f,g)$
is an $R[u]$-linear combination of $f$ and $g$. (This follows from
our definitions above and the analogous result for the usual
resultant.) In particular, if $R$ is a field and $\res_u^{n,m}(f,g)$
is a nonzero constant, then $f$ and $g$ have no common roots in $R$.

We use $\Sym(S)$ to denote the symmetric group on the set $S$.

\subsection{Further Preliminaries for the Proof of Theorem \ref{thm:main}}
Since the case $n=1$ of Theorem \ref{thm:main} is trivial, we always
suppose that $n \geq 2$. We also suppose the following setup:
\begin{alignat*}{1}
f_1, \dots, f_r &\qquad\text{pairwise nonassociated irreducible
univariate polynomials over $\F_q$,} \\
d_1, \dots, d_r &\qquad \text{degrees of $f_1, \dots, f_r$
respectively,} \\
\theta_1, \dots, \theta_r &\qquad \text{fixed roots of $f_1, \dots,
f_r$, respectively, from $\overline{\F}_q$,} \\
\theta_i^{(j)} &\qquad \text{$j$th conjugate of $\theta_i$ with
respect to Frobenius, i.e., $\theta_i^{(j)}:= \theta_i^{q^j}$.}
\end{alignat*}

If $h$ is a fixed polynomial of degree $n\geq 2$ over $\F_q$, we
define the function fields $K_{i,j}/\F_q, L_{i,j}/\F_q$ and
$M_i/\F_q$ (for $1 \leq i \leq r, 1 \leq j \leq d_i$) as follows,
suppressing in our notation the dependence on $h$:
\begin{alignat*}{1}
K_{i,j} &\qquad\text{field obtained by adjoining a fixed root of
$h(T) - u - \theta_i^{(j)}$ to $\F_{q^{d_i}}(u)$,} \\
L_{i,j} &\qquad \text{Galois closure of $K_{i,j}$ over
$\F_{q^{d_i}}(u)$,} \\
M_i &\qquad \text{compositum of the fields $L_{i,j}$ for $j=1, 2
\dots, d_i$.}
\end{alignat*}

We let $D$ be the least common multiple of $d_1, \dots, d_r$ and
denote with a tilde the corresponding fields obtained by extending
the constant field by $\F_{q^D}$. (That is, we set
$\widetilde{K}_{i,j}:= K_{i,j}\F_{q^D}, \widetilde{L}_{i,j}:=
L_{i,j} \F_{q^D}$ and $\widetilde{M}_i:=M_i \F_{q^D}$.) Finally, we
let $\widetilde{M}$ denote the compositum of $\widetilde{M}_1,
\dots, \widetilde{M}_r$. The inclusion relations between these
fields are illustrated in Figures \ref{fig:tower} and
\ref{fig:bigtower}.


\begin{figure}\[
\xy
 (0,0)*+{\F_{q^{d_i}}(u)};(16,12)*+{K_{i,d_i}}**\dir{-};
(-16,12)*+{K_{i,1}};(-32,24)*+{L_{i,1}}**\dir{-};
(16,12)*+{K_{i,d_i}};(32,24)*+{L_{i,d_i}}**\dir{-};
(0,-13)*+{\F_{q}(u)};(0,0)*+{\F_{q^{d_i}}(u)}**\dir{-};
(-16,12)*+{K_{i,1}}**\dir{-};(0,0)*+{\F_{q^{d_i}}(u)};(16,12)*+{K_{i,d_i}}**\dir{-};(0,0)*+{\F_{q^{d_i}}(u)};
(-32,24)*+{L_{i,1}};(0,40)*+{M_i}**\dir{-};
(32,24)*+{L_{i,d_i}};(0,40)*+{M_i}**\dir{-};
(0,0)*+{\F_{q^{d_i}}(u)};(-16,24)*+{\dots}**\dir{--};
(0,0)*+{\F_{q^{d_i}}(u)};(16,24)*+{\dots}**\dir{--};
(-16,24)*+{\dots};(0,40)*+{M_i}**\dir{--};(0,0)*+{\F_{q^{d_i}}(u)};(0,12)*+{K_{i,j}}**\dir{--};
(0,12)*+{K_{i,j}};(0,24)*+{L_{i,j}}**\dir{--};
(0,24)*+{L_{i,j}};(0,40)*+{M_i}**\dir{--};
(16,24)*+{\dots};(0,40)*+{M_i}**\dir{--};
\endxy
\]
\caption{Tower of fields illustrating the inclusion relations
between $\F_{q}(u), \F_{q^{d_i}}(u)$, the $K_{i,j}$, the $L_{i,j}$
and $M_i$.\label{fig:tower} }
\end{figure}


\begin{figure}
\[
\xy (0,0)*+{};(0,-10)*+{\F_{q}(u)}**\dir{-};
(0,0)*+{\F_{q^{D}}(u)};(-15,5)*+{\widetilde{K}_{1,1}}**\dir{-};
(-15,5)*+{\widetilde{K}_{1,1}};(-30,10)*+{\widetilde{L}_{1,1}}**\dir{-};
(0,0)*+{\F_{q^{D}}(u)};(-20,0)*+{\widetilde{K}_{1,j}}**\dir{--};
(-20,0)*+{\widetilde{K}_{1,j}};(-40,0)*+{\widetilde{L}_{1,j}}**\dir{--};
(0,0)*+{\F_{q^{D}}(u)};(-15,-5)*+{\widetilde{K}_{1,d_1}}**\dir{-};
(-15,-5)*+{\widetilde{K}_{1,d_1}};(-30,-10)*+{\widetilde{L}_{1,d_1}}**\dir{-};
(-60,0)*+{\widetilde{M}_{1}};(-30,-10)*+{\widetilde{L}_{1,d_1}}**\dir{-};
(-60,0)*+{\widetilde{M}_{1}};(-40,0)*+{\widetilde{L}_{1,j}}**\dir{--};
(-60,0)*+{\widetilde{M}_{1}};(-30,10)*+{\widetilde{L}_{1,1}}**\dir{-};
(0,0)*+{\F_{q^{D}}(u)};(15,5)*+{\widetilde{K}_{r,1}}**\dir{-};
(15,5)*+{\widetilde{K}_{r,1}};(30,10)*+{\widetilde{L}_{r,1}}**\dir{-};
(0,0)*+{\F_{q^{D}}(u)};(20,0)*+{\widetilde{K}_{r,j}}**\dir{--};
(20,0)*+{\widetilde{K}_{r,j}};(40,0)*+{\widetilde{L}_{r,j}}**\dir{--};
(0,0)*+{\F_{q^{D}}(u)};(15,-5)*+{\widetilde{K}_{r,d_r}}**\dir{-};
(15,-5)*+{\widetilde{K}_{r,d_r}};(30,-10)*+{\widetilde{L}_{r,d_r}}**\dir{-};
(60,0)*+{\widetilde{M}_{r}};(30,-10)*+{\widetilde{L}_{r,d_r}}**\dir{-};
(60,0)*+{\widetilde{M}_{r}};(40,0)*+{\widetilde{L}_{r,j}}**\dir{--};
(60,0)*+{\widetilde{M}_{r}};(30,10)*+{\widetilde{L}_{r,1}}**\dir{-};
(0,0)*+{\F_{q^{D}}(u)};(10,15)*+{\widetilde{K}_{i,d_i}}**\dir{-};
(10,15)*+{\widetilde{K}_{i,d_i}};(20,30)*+{\widetilde{L}_{i,d_i}}**\dir{-};
(0,0)*+{\F_{q^{D}}(u)};(0,15)*+{\widetilde{K}_{i,j}}**\dir{--};
(0,15)*+{\widetilde{K}_{i,j}};(0,30)*+{\widetilde{L}_{i,j}}**\dir{--};
(0,0)*+{\F_{q^{D}}(u)};(-10,15)*+{\widetilde{K}_{i,1}}**\dir{-};
(-10,15)*+{\widetilde{K}_{i,1}};(-20,30)*+{\widetilde{L}_{i,1}}**\dir{-};
(-20,30)*+{\widetilde{L}_{i,1}};(0,50)*+{\widetilde{M}_{i}}**\dir{-};
(20,30)*+{\widetilde{L}_{i,d_i}};(0,50)*+{\widetilde{M}_{i}}**\dir{-};
(0,30)*+{\widetilde{L}_{i,j}};(0,50)*+{\widetilde{M}_{i}}**\dir{--};
(0,80)*+{\widetilde{M}};(0,50)*+{\widetilde{M}_{i}}**\dir{-};
(0,80)*+{\widetilde{M}};(-60,0)*+{\widetilde{M}_{1}}**\crv{(-40,60)};
(0,80)*+{\widetilde{M}};(60,0)*+{\widetilde{M}_{r}}**\crv{(40,60)};
\endxy\]
\caption{Field diagram illustrating the inclusion relations between
$\F_{q}(u), \F_{q^{D}}(u)$, the $\widetilde{K}_{i,j}$, the
$\widetilde{L}_{i,j}$, $\widetilde{M}_i$ and $\widetilde{M}$. Here
moving to a larger field is signified by moving outward from
$\F_q(u)$. \label{fig:bigtower}}
\end{figure}



\begin{lem}\label{lem:hnon} Assume that $h(T)$ is a polynomial of degree $n\geq 2$ over $\F_q$
which is not a polynomial in $T^p$, where $p$ is the characteristic
of $\F_q$. Then the extensions $M_i/\F_q(u)$ are Galois for each
$i=1, 2, \dots, r$. The same assertion holds for the extensions
$\widetilde{M}_i/\F_q(u)$ and $\widetilde{M}/\F_q(u)$.
\end{lem}
\begin{proof} Observe that $M_i$ is the splitting field over
$\F_q(u)$ of $f_i(h(T)-u)$, so that the first half of the lemma
follows immediately once we show that the irreducible factors of
$f_i(h(T)-u)$ are separable over $\F_q(u)$. Moving to the finite
extension $\F_{q^{d_i}}(u)$ of $\F_q(u)$ we have
\[ f_i(h(T)-u) = \prod_{j=1}^{d_i}(h(T) - u - \theta_i^{(j)}). \]
The $d_i$ factors on the right-hand side are pairwise coprime (in
$\overline{\F_q(u)}[T]$), so that it suffices to verify that each
factor $h(T) - u - \theta_i^{(j)}$ has no repeated roots. Any such
repeated root is also a root of $h'(T)$. But our hypothesis on $h$
ensures that $h'$ is not identically zero, so each root of $h'(T)$
is algebraic over $\F_q$, while $h(T) - u - \theta_i^{(j)}$ has no
roots algebraic over $\F_q$.

The second half of the lemma is a consequence of the first. Indeed,
since $\F_{q^D}(u)/\F_q(u)$ is Galois, what we have just proved
implies that $\widetilde{M}_i = M_i \F_{q^D} = M_i \F_{q^D}(u)$ is
also Galois over $\F_q(u)$, and thus so is the compositum of the
$\widetilde{M}_i$.
\end{proof}

The groups $\Gal(\widetilde{M}/\F_{q}(u))$ and $\Gal(M_i/\F_q(u))$
will play an important role and so we study them in some detail. Let
$S_{i,j}$ denote the full set of roots of $h(T) - u -
\theta_{i}^{(j)}$ (thus $S_{i,j}$ depends only on $j\bmod{d_i}$). We
begin by observing that under the hypothesis of Lemma
\ref{lem:hnon}, which assures that the extensions appearing below
are Galois, we have for each $k=1, 2, \dots, r$ a commutative
diagram
\begin{equation}\label{eq:cd}
\begin{CD}
\Gal(\widetilde{M}/\F_q(u)) @>\iota_1>> \Gal(\F_{q^D}/\F_q)\times
\prod_{i=1}^{r}\mathrm{Sym}(\cup_{j=1}^{d_i}S_{i,j})\\
@V \sigma \mapsto \sigma|_{M_k} VV @V\pi VV \\
\Gal(M_k/\F_q(u)) @>\iota_2 >> \Gal(\F_{q^{d_k}}/\F_q)\times
\mathrm{Sym}(\cup_{j=1}^{d_k}S_{k,j})
\end{CD}.
\end{equation} Here the maps $\iota_1, \iota_2$ are given by
\begin{align*}
 \iota_1\colon \sigma &\mapsto (\sigma|_{\F_{q^D}}, \sigma|_{\cup_{j=1}^{d_1} S_{1,j}}, \dots, \sigma|_{\cup_{j=1}^{d_r}
 S_{r,j}}),\\
\iota_2\colon \sigma &\mapsto (\sigma|_{\F_{q^{d_k}}},
\sigma|_{\cup_{j=1}^{d_k} S_{k,j}}),
\end{align*}
and
\[ \pi\colon (\tau, \sigma_1, \dots, \sigma_r) \mapsto (\tau|_{\F_{q^{d_k}}}, \sigma_k). \]
Note that $\iota_1$ and $\iota_2$ are embeddings while $\pi$ is a
surjection.

The remainder of this section is devoted to an explicit description
of the images of $\iota_1$ and $\iota_2$ under a mild restriction on
$h$. This characterization is obtained under the following two
hypotheses:
\begin{equation}\label{eq:disccond}
 \disc_u^{n-1}\disc_{T}^n (h(T)-u - \theta_i^{(j)}) \neq 0 \quad \text{for all}\quad 1 \leq i \leq r, \quad 1 \leq j\leq
 d_i,
\end{equation}
and
\begin{multline}\label{eq:rescond}
 \res_{u}^{n-1,n-1}\left(\disc_{T}^n (h(T)-u -
\theta_i^{(j)}),\disc_{T}^n (h(T)-u - \theta_{i'}^{(j')})\right) \neq 0 \\
\text{whenever $i, i', j, j'$ are as above and $(i,j) \neq (i',
j')$.}
\end{multline}
(Note that from \eqref{eq:disccond} we have immediately that $h$ is
not a polynomial in $T^p$.) That together \eqref{eq:disccond} and
\eqref{eq:rescond} impose only a mild restriction on $h$ is borne
out by the following lemma, which we prove in \S\ref{sec:proofs}:

\begin{lem}\label{lem:few} Let $h(T)$ range over the polynomials of the
form $T^n + a_{n-1} T^{n-1} + \dots + a_1 T$, with all coefficients
$a_i$ belonging to $\F_q$. Assume that $q$ is prime to $2n$. Then
both of the following hold:
\begin{enumerate}\item The number of such $h$ for which
\eqref{eq:disccond} fails is bounded above by
\begin{equation}\label{eq:bounds}
 (2n-1)(2n-3) q^{n-2}. \end{equation}
 \item For any fixed pairs of indices $(i,j) \neq (i',j')$, the same bound holds for the
number of such $h$ which fail to satisfy \eqref{eq:rescond}.
\end{enumerate}
Consequently, for all but at most
\[ 4n^2\left(1 + \binom{d_1 + \dots + d_r}{2}\right) q^{n-2} \]
values of $h$ as above, both \eqref{eq:disccond} and
\eqref{eq:rescond} hold for all distinct pairs of indices $(i, j)$
and $(i', j')$.
\end{lem}

We now present the promised descriptions of the images of $\iota_1$
and $\iota_2$, beginning with $\iota_2$:

\begin{lem}\label{lem:iota2} Let $n \geq 2$. Assume that the characteristic of $\F_q$ is prime to $2n$.
Then if $h(T)$ has the form
\[ h(T) = T^n + a_{n-1} T^{n-1} + \dots + a_1 T, \quad \text{with each $a_i \in \F_q$}, \]
and $h(T)$ satisfies both \eqref{eq:disccond} and
\eqref{eq:rescond}, then all of the following hold:
\begin{enumerate}
\item The $L_{i,j}$ are Galois over $\F_{q^{d_i}}(u)$ with
Galois group $\Sym(S_{i,j})$ for each $1 \leq i \leq r, 1 \leq j
\leq d_i$.
\item For every $1 \leq i \leq r, 1 \leq j \leq d_i$, the field $L_{i,j}$
is linearly disjoint from the compositum of all other fields $L_{i,
j'}$ with $1 \leq j' \neq j \leq d_i$.
\item $\F_{q^{d_i}}$ is the full field of constants of $M_i/\F_{q^{d_i}}$.
\item The extension $M_i/\F_{q^{d_i}}(u)$ is Galois with
\[ \Gal(M_i/\F_{q^{d_i}}(u)) \cong \prod_{j=1}^{d_i}
\Gal(L_{i,j}/\F_{q^{d_i}}(u)) \cong \prod_{j=1}^{d_i} \Sym(S_{i,j}),
\] the first isomorphism being induced by restriction in each
component.
\item Fix $1 \leq i \leq r$. Let $\Frob$ denote the $q$th power map, so that
$\Frob$ generates $\Gal(\F_{q^{d_i}}/\F_q)$. The image of $\iota_2$
consists of all pairs $(\Frob^k, \sigma)$ which obey the following
compatibility condition:
\[ \sigma(S_{i,j}) \subset \sigma(S_{i,j+k}). \]
\end{enumerate}
\end{lem}

A similar lemma characterizes the image of $\iota_1$:

\begin{lem}\label{lem:iota1} Let $n \geq 2$. Assume that the characteristic of $\F_q$ is prime to $2n$.
Then if $h(T)$ has the form
\[ h(T) = T^n + a_{n-1} T^{n-1} + \dots + a_1 T, \quad \text{with each $a_i \in \F_q$}, \]
and $h(T)$ satisfies both \eqref{eq:disccond} and
\eqref{eq:rescond}, then all of the following hold:
\begin{enumerate}
\item The fields $\widetilde{L}_{i,j}$ are Galois over $\F_{q^{D}}(u)$ with
Galois group $\Sym(S_{i,j})$ for each $1 \leq i \leq r, 1 \leq j
\leq d_i$.
\item For every $1 \leq i \leq r, 1 \leq j \leq d_i$, the field $\widetilde{L}_{i,j}$
is linearly disjoint from the compositum of all other fields
$\widetilde{L}_{i', j'}$ with $1 \leq i' \leq r,  1 \leq j' \leq
d_{i'}$ and $(i,j) \neq (i', j')$.
\item $\F_{q^{D}}$ is the full field of constants of $\widetilde{M}$.
\item The image of $\iota_2$ consists of all pairs $(\Frob^k, \sigma)$ which obey the
compatibility condition
\[ \sigma(S_{i,j}) \subset \sigma(S_{i,j+k}) \qquad \text{for every $i=1, 2, \dots, r$}. \]
\end{enumerate}
\end{lem}

The proofs of Lemmas \ref{lem:few}, \ref{lem:iota2}, and
\ref{lem:iota1} are deferred to the next section. The curious reader
may jump directly to the proof of Theorem \ref{thm:main} in
\S\ref{sec:thm}.

\section{Proofs of Lemmas \ref{lem:few}, \ref{lem:iota2},
and \ref{lem:iota1}}\label{sec:proofs}
\subsection{Proof of Lemma \ref{lem:few}}
The proof of Lemma \ref{lem:few} rests on the following elementary
bound for the number of affine zeros of a polynomial:

\begin{lem}\label{lem:pbound} Let $E/\F_q$ be an arbitrary field extension
and let $P(T_1,\dots, T_m)$ be a nonzero polynomial in $m$ variables
over $E$ with total degree bounded by $d$. Then there are at most $d
q^{m-1}$ solutions to $P(x_1, \dots, x_m) = 0$ in $\F_q^{m}$.
\end{lem}

This lemma is well-known in the case when $E = \F_q$ (see, e.g.,
\cite[Theorem 6.13]{ln97}), and the general case reduces to this one
upon writing the coefficients of $P$ with respect to an $\F_q$-basis
of $E$.

Our computations also require the following evaluation of the
discriminants of certain trinomials (cf. \cite[Exercise
4.5.4]{esmonde-murty99}):

\begin{lem}\label{lem:discriminants} Let $R$ be any integral domain, and let $a$ and $b$ be
any elements of $R$. Then
\[ \disc_{T}(T^n + aT + b) = (-1)^{\binom{n}{2}} (n^n b^{n-1} + (-1)^{n-1} (n-1)^{n-1} a^n).\]
\end{lem}

\begin{proof}[Proof of Lemma \ref{lem:few}(i)] For every pair of $i$ and $j$
with $1 \leq i \leq r$ and $1 \leq j \leq d_i$,  we have
\begin{equation}\label{eq:multex}
\disc_u^{n-1}\disc_{T}^n (h(T)-u - \theta_i^{(j)}) =
\disc_u^{n-1}\disc_{T}^n (h(T)-u);
\end{equation}
indeed, the $T$-discriminant on the left-hand side differs from the
one on the right only in that $u$ is replaced by $u-\theta_i^{(j)}$,
and such a shift leaves the outer $u$-discriminant unaffected.

Define a polynomial $\hat{P}$ with integer coefficients in the $n-1$
indeterminates $T_1, \dots, T_{n-1}$ by
\begin{equation}\label{eq:hatpdef}
\hat{P}(T_1, \dots, T_{n-1}) := \disc_u^{n-1} \disc_T^{n}(T^n + T_{n-1} T^{n-1} + \dots + T_{1} T
-u).
\end{equation}
(Note that $T$ and $u$ are successively eliminated by the right-hand
discriminants, so that only the indeterminates $T_1, \dots, T_{n-1}$
remain.) We claim that if $q$ is prime to $2n$, then $\hat{P}$ does
not reduce to the zero polynomial when considered over $\F_q$. This
suffices to prove \eqref{eq:bounds}. To see why, observe (from the
definition of the discriminant in terms of the determinant of the
$(2n-1)\times (2n-1)$ Sylvester matrix) that the inner
$T$-discriminant on the right of \eqref{eq:hatpdef} is a polynomial
in $u$ of degree at most $n-1$, each coefficient of which is a
polynomial in $T_1, \dots, T_{n-1}$ of total degree bounded by
$2n-1$. These coefficients determine the entries of the $(2n-3)
\times (2n-3)$ determinant used to compute $\hat{P}$, whence
$\hat{P}$ has total degree at most $(2n-1)(2n-3)$ in $T_1, \dots,
T_{n-1}$. The desired bound \eqref{eq:bounds} on the number of $h$
which fail to satisfy \eqref{eq:disccond} now follows from Lemma
\ref{lem:few}.

It remains to prove our claim that $\hat{P}$ is nonvanishing when
considered over $\F_q$. This is easiest if we adopt the further
assumption that the characteristic $p$ of $\F_q$ is prime to $n-1$.
Indeed, successive application of Lemma \ref{lem:discriminants}
shows
\begin{align*}
 \hat{P}(1,0,\dots,0) &= \disc_u^{n-1} \disc_T^{n}(T^n +T-u) \\
 &= \disc_u^{n-1} \left((-1)^{\binom{n}{2}}\left(n^n (-u)^{n-1} + (-1)^{n-1}(n-1)^{n-1}\right)\right) \\
 &= \disc_u^{n-1} (n^n u^{n-1} + (n-1)^{n-1}) = \pm
 (n-1)^{(n-1)^2} n^{n(n-2)},
\end{align*}
which is nonzero under this additional hypothesis.

We therefore suppose that $p$ divides $n-1$. In this case we
consider
\begin{align*}
 \hat{P}(1,1,\dots,1) &= \disc_u^{n-1} \disc_{T}^n(T^n + T^{n-1} + \dots
 + T - u).
\end{align*}
To understand the inner discriminant, note that
\[ (T-1)(T^n + T^{n-1} + \dots + T - u)= T^{n+1} - T - (T-1)u. \]
By Lemma \ref{lem:discriminants}, the $T$-discriminant of the
right-hand polynomial is given explicitly by
\begin{equation}\label{eq:preq} (-1)^{\binom{n+1}{2}}\left((n+1)^{n+1} u^n - n^n (u+1)^{n+1}\right).
\end{equation} We can relate this to the discriminant we are after
by using the relations
\begin{multline*}  \disc_T((T-1)
(T^n + T^{n-1} + \dots
 + T - u)) = \\\pm\left((T^n + T^{n-1} + \dots + T-u)|_{T=1}\right)^2 \disc_T(T^n + T^{n-1}+\dots
 + T - u) =\\
\pm (n-u)^2 \disc_T(T^n + T^{n-1}+ \dots
 + T - u).
 \end{multline*} Piecing this all together we obtain
\[ \hat{P}(1,1,\dots,1) = \disc_u^{n-1}\left(\frac{(n+1)^{n+1} u^n
- n^n (u+1)^{n+1}}{(u-n)^2}\right). \]

Let $Q(u)$ denote the polynomial in $u$ appearing in the argument of
$\disc_u$ here, so that $Q$ has degree $n-1$ in $u$. If
$\hat{P}(1,1,\dots,1)$ vanishes, then $Q$ has a multiple root, which
is necessarily also a multiple root of \eqref{eq:preq}. One computes
easily that unless $p$ divides $n+1$, the only common root of
\eqref{eq:preq} and its derivative is $u=n$. If $u=n$ is a multiple
root of $Q$, then it must be a root of multiplicity at least $4$ of
\eqref{eq:preq}, which forces the second derivative of
\eqref{eq:preq} to vanish at $u=n$. But this second derivative is
given by \begin{multline*} (-1)^{\binom{n+1}{2}} \left((n+1)^{n+1} n
(n-1) n^{n-2} - n^{n+1} (n+1) (n+1)^{n-1}\right) =
\\ (-1)^{\binom{n+1}{2}+1}n^{n-1} (n+1)^n. \end{multline*} Since the characteristic
$p$ is prime to $n$, this can only vanish if $p$ divides $n+1$. So
we are forced to the conclusion that $\hat{P}(1,\dots,1)$ is
nonvanishing except possibly if $p$ divides $n+1$. However, $p$
divides $n-1$ in the case we are considering, so that $p$ can divide
$n+1$ only if $p=2$, which is excluded.
\end{proof}


\begin{proof}[Proof of Lemma \ref{lem:few}(ii)] We proceed as in the
proof of Lemma \ref{lem:few}(i). Write $h(T) = T^n + a_{n-1}T^{n-1}+
\dots + a_1 T$ as usual. Fix pairs $(i,j)$ and $(i', j')$ with
$(i,j) \neq (i', j')$ and set
\[ P(a_1, \dots, a_{n-1}):= \res_{u}^{n-1,n-1}\left(\disc_{T}^n{(h(T)-u-\theta_{i}^{(j)})},
\disc_{T}^n{(h(T)-u-\theta_{i'}^{(j')})}\right). \] Arguing as in
Lemma \ref{lem:few}(i), we see that there is some polynomial
$\hat{P}(T_1, \dots, T_{n-1})$ over $\overline{\F}_q$ of degree at
most $(2n-1)(2n-3)$ for which
\[ P(a_1, \dots, a_{n-1}) = \hat{P}(a_1, \dots, a_{n-1}) \quad\text{for all $a_1, \dots, a_{n-1} \in \F_q$}. \]
Then \eqref{eq:rescond} is satisfied (for the fixed pairs $(i,j)$
and $(i',j')$) as long as $\hat{P}$ is nonvanishing. This
nonvanishing is easily checked: indeed,
\begin{align*}
 \hat{P}(0,\dots,0)&= \res_{u}^{n-1,n-1}(\disc_T(T^n - u
 -\theta_{i}^{(j)}),\disc_T(T^n-u-\theta_{i'}^{(j')})) \\
&= \res_{u}^{n-1,n-1}(\disc_T(T^n
-u),\disc_T(T^n-u+\theta_{i}^{(j)}-\theta_{i'}^{(j')}))\\ &=
(-1)^{n+1} n^{n(2n-2)}
(\theta_{i}^{(j)}-\theta_{i'}^{(j')})^{(n-1)^2} \neq 0.\end{align*}
Lemma \ref{lem:pbound} now implies that $\hat{P}$ has at most
$(2n-1)(2n-3)q^{n-2}$ zeros in $\F_q^{n-1}$, finishing the proof.
\end{proof}



\subsection{Proofs of Lemmas \ref{lem:iota2} and \ref{lem:iota1}}
Our fundamental tool is the following criterion of Birch \&
Swinnerton-Dyer \cite{bs59} for certain polynomials to have the full
symmetric group as their Galois group. We state their result in an
alternative form attributed by the same authors to Davenport:

\begin{bsd}\label{lem:bsd} Let $h$ be a
polynomial of degree $n\geq 2$ with coefficients from a finite field
$F$ whose characteristic is prime to $n$. Suppose that with $u$ an
indeterminate over $F$, we have
\begin{equation}\label{eq:bsdcriterion}
\disc_u^{n-1}\disc_{T}^n (h(T)-u) \neq 0.
\end{equation}
Then the Galois group of $h(T)-u$ over the rational function field
$\overline{F}(u)$ is the full symmetric group on the $n$ roots of
$h(T)-u$. Consequently, if $E$ is any algebraic extension of $F$,
then the Galois group of $h(T)-u$ over $E(u)$ is also the full
symmetric group.
\end{bsd}

\begin{proof}[Proof of Lemmas \ref{lem:iota2}(i) and \ref{lem:iota1}(i)]
Suppose that $h$ satisfies both conditions \eqref{eq:disccond} and
\eqref{eq:rescond}. Then part (i) of Lemma \ref{lem:iota2} is
immediate from the Birch \& Swinnerton-Dyer criterion. Since
$\widetilde{L}_{i,j}$ is the splitting field of $h(T) - u -
\theta_i^{(j)}$ over $\F_{q^{D}}$, the same argument also
establishes Lemma \ref{lem:iota1}(i).\end{proof}

To continue we require two more technical tools. The first is a
lemma of Hayes appearing in an alternative proof of the Birch \&
Swinnerton-Dyer criterion:

\begin{lem}[(Hayes)]\label{lem:hayes} Let $h$ be a polynomial of degree $n \geq 2$ over
the finite field $\F_q$ which satisfies the hypotheses of the Birch
\& Swinnerton-Dyer criterion with $F = \F_q$. Let $L$ be the
splitting field of $h(T)-u$ over $\overline{\F}_q(u)$. Let
$P_{\infty}$ be the prime of $\overline{\F}_{q}(u)$ corresponding to
the $(1/u)$-adic valuation on $\overline{\F}_q[1/u]$, and let $P$ be
any prime of $L$ lying above above $P_{\infty}$. Then $e(P |
P_{\infty}) = n$, where $e(P|P_{\infty})$ denotes the ramification
index of $P$ over $P_{\infty}$.
\end{lem}
Hayes proves this explicitly only in the case $h = T^n + T-u$ (see
\cite[Proof of Lemma 1]{hayes73}), but as he remarks the arguments
extend easily to the general case. It is necessary for us to also
understand the ramification of $P_{\infty}$ in certain extensions of
the fields appearing in Hayes's lemma; for this we appeal to the
following result (\cite[Proposition III.8.9]{stichtenoth93}):

\begin{abh} Let $F'/F$ be a finite separable extension of function
fields. Suppose that $F' = F_1 F_2$ is the compositum of two
intermediate fields $F \subset F_1, F_2 \subset F'$. Let $P$ be a
prime of $F$ and $P'$ a prime of $F'$ lying above $P$. With $P_i :=
P' \cap F_i$ for $i=1$ and $2$, assume that at least one of the
extensions $P_1/P$ or $P_2/P$ is tame (i.e., that $e(P_i/P)$ is
relatively prime to the characteristic of $F$). Then
\[ e(P'/P) = \lcm[e(P_1/P), e(P_2/P)]. \]
In particular, if both $P_1/P$ and $P_2/P$ are tamely ramified, then
so is $P'/P$.
\end{abh}


\begin{proof}[Proof of Lemmas \ref{lem:iota2}(ii) and \ref{lem:iota1}(ii)]
Define the constant field extensions \[ \hat{K}_{i,j} := K_{i,j}
\overline{\F}_q, \quad \hat{L}_{i,j} := L_{i,j}
\overline{\F}_q,\quad \text{and}\quad \hat{M}_{i} := M_i
\overline{\F}_q.\] Thus $\hat{L}_{i,j}$ is the splitting field of
$h(T) - u - \theta_i^{(j)}$ over $\overline{\F}_q$. To prove Lemma
\ref{lem:iota2}(ii), it suffices to show that for each fixed $i$,
\begin{equation}\label{eq:disjoint}
\hat{L}_{i,j} \text{ is linearly disjoint from the compositum of
$\hat{L}_{i,j'}$ for $1 \leq j' \neq j \leq d_i$.}
\end{equation} Indeed, once \eqref{eq:disjoint} is known, we may
deduce that
\[ \Gal(\hat{M}_i/\overline{\F}_q(u)) \cong \Gal(\hat{L}_{i,1}/\overline{\F}_q(u)) \times \dots
\times \Gal(\hat{L}_{i,d_i}/\overline{\F}_q(u)). \] By the Birch \&
Swinnerton-Dyer criterion the right-hand Galois groups each have
size $n!$, so that the left-hand Galois group has size $n!^{d_i}$.
But the left-hand Galois group injects (via restriction) into
$\Gal(M_i/\F_{q^{d_i}}(u))$, and degree counting shows that this
injection must be an isomorphism; thus
\begin{multline*} [M_i: \F_{q^{d_i}}(u)] = [L_{i,1} L_{i,2} \cdots L_{i, d_i}: \F_{q^{d_i}}(u)] = \\
[L_{i,1}: \F_{q^{d_i}}(u)] [L_{i,2}: \F_{q^{d_i}}(u)] \cdots
[L_{i,d_i}: \F_{q^{d_i}}(u)],
\end{multline*}
which implies Lemma \ref{lem:iota2}(ii).

To prove \eqref{eq:disjoint}, consider the intersection $N$ of
$\hat{L}_{i,j}$ with the compositum of the fields $\hat{L}_{i,j'}$
for $1 \leq j \neq j' \leq d_i$. The only primes of
$\overline{\F}_q(u)$ that can ramify in $N$ ramify in both
$\hat{K}_{i,j}$ and some $\hat{K}_{i,j'}$ with $1 \leq j \neq j'
\leq d_i$. But by \eqref{eq:rescond}, the polynomials
\[ \disc_{T}^{n}(h(T) - u - \theta_i^{(j)}) \quad \text{and} \quad
\disc_{T}^{n}(h(T) - u - \theta_i^{(j')}) \quad \text{have no common
roots},
\] and so the only prime that can possibly ramify in both extensions is $P_{\infty}$. By
Hayes's Lemma \ref{lem:hayes} and repeated application of
Abhyankar's Lemma, $P_{\infty}$ is tamely ramified in
$\hat{L}_{i,j}$ and hence also in $N$. (Here we again use our
hypothesis that $q$ is prime to $n$.) Thus $N$ is a finite, tamely
ramified geometric extension of $\overline{\F}_q(u)$ unramified
except possibly at primes above the degree $1$ prime $P_{\infty}$.
It follows that $N= \overline{\F}_q(u)$ (this is an immediate
consequence of the Riemann-Hurwitz genus formula; see, e.g.,
\cite[p.460]{hayes73} or \cite[Exercise 6, p.99]{rosen02}). This
proves \eqref{eq:disjoint} and together with the above argument
completes the proof of Lemma \ref{lem:iota2}(ii).

The proof of Lemma \ref{lem:iota1}(ii) is nearly identical but is
based instead on the claim that
\begin{equation}\label{eq:disjoint2}
\hat{L}_{i,j} \text{ is linearly disjoint from the compositum of
$\hat{L}_{i,j}$ for $(i,j) \neq (i', j')$;}
\end{equation}
we omit the details.\end{proof}

\begin{proof}[Proof of Lemmas \ref{lem:iota2}(iii) and
\ref{lem:iota1}(iii)] In the course of proving Lemma
\ref{lem:iota2}(ii), we showed that restriction induces an
isomorphism
\[ \Gal(\hat{M}_i/\overline{\F}_q(u)) \cong \Gal(M_i/\F_{q^{d_i}}(u)).\]
If $\alpha \in M_i \cap \overline{\F}_q$, then $\alpha$ is fixed by
every element of the left-hand Galois group appearing above, and so
must be fixed by all elements of the right-hand Galois group. But
this forces $\alpha$ to lie in the field of rational functions
$\F_{q^{d_i}}(u)$. Since $\alpha$ is algebraic over $\F_q$, it must
belong to $\F_{q^{d_i}}$. So $\F_{q^{d_i}}$ is the full field of
constants of $M_i$. Lemma \ref{lem:iota1}(iii) can be proved
similarly, using that restriction induces an isomorphism
$\Gal(\widetilde{M} \overline{\F}_q/\overline{\F}_q(u)) \cong
\Gal(\widetilde{M}/\F_{q^{D}}(u))$.\end{proof}

\begin{proof}[Proof of Lemma \ref{lem:iota2}(iv)] This is immediate from parts (i) and (ii) of Lemma \ref{lem:iota2}.
\end{proof}

\begin{proof}[Proof of Lemma \ref{lem:iota2}(v) and Lemma
\ref{lem:iota1}(iv)] Suppose that $\sigma \in
\Gal(M_i/\F_{q^{d_i}}(u))$ satisfies $\sigma|_{\F_{q^{d_i}}} =
\Frob^k$. Then $\sigma$ takes $\theta_i^{(j)}$ to $\theta_i^{(j+k)}$
and so takes every root of $h(T)-u-\theta_i^{(j)}$ to a root of
$h(T) - u - \theta_i^{(j+k)}$. It follows that the image of
$\iota_2$ is contained within the set of elements obeying the
compatibility condition specified in Lemma \ref{lem:iota2}(v). A
straightforward counting argument shows that there are $d_i
n!^{d_i}$ such elements of $\Gal(\F_{q^{d_i}}/\F_{q}) \times
\Sym(\cup_{j=1}^{d_i}S_{i,j})$. On the other hand, we know that
$M_i/\F_q(u)$ is Galois of degree $[M_i: \F_q(u)] =[M_i:
\F_{q^{d_i}}(u)][\F_{q^{d_i}}(u):\F_q(u)] = d_i n!^{d_i}$. Since
$\iota_2$ is injective, it follows that the image of $\iota_2$ must
coincide with the set specified in (v).

A similar argument establishes Lemma \ref{lem:iota1}(iv): in that
case $\widetilde{M}$ is Galois over $\F_q(u)$ of degree $D n!^{d_1+
\dots + d_r}$, and this degree coincides with the number of elements
obeying the compatibility condition of Lemma \ref{lem:iota1}(iv).
\end{proof}

\section{Proof of Theorem \ref{thm:main}}\label{sec:thm}

Throughout this section $f_1(T), \dots, f_r(T)$ denote pairwise
nonassociated irreducible polynomials of respective degrees $d_1,
\dots, d_r$ over $\F_q$ and $h(T) = T^n + a_{n-1}T^{n-1} + \dots +
a_1 T$ denotes a monic polynomial of degree $n \geq 2$ without
constant term satisfying conditions \eqref{eq:disccond} and
\eqref{eq:rescond}.

Our plan is to use the Chebotarev density theorem to estimate, for
each individual $h(T)$, the number of $a \in \F_q$ for which all of
the specializations $f_i(h(T)-a)$ are irreducible. We begin by
recalling the following well-known lemma (see, e.g., \cite[pp.
408-409]{cohen89}):

\begin{lem}\label{lem:subst} Let $f$ be an irreducible polynomial of degree $d$
over $\F_q$ and let $\theta$ be a root of $f$ from the extension
$\F_{q^d}$. Let $p(T)$ be a nonconstant polynomial over $\F_q$. Then
$f(p(T))$ is irreducible over $\F_q$ if and only if $p(T) - \theta$
is irreducible over $\F_{q^d}$.
\end{lem}

The next result explains how the Chebotarev density theorem enters
the picture:

\begin{lem}\label{lem:class} There is a conjugacy class $\C$ of $\Gal(\widetilde{M}/\F_{q}(u))$ with the following
property: If $a$ is an element of $\F_q$ which is not a zero of any
of the polynomials \begin{equation}\label{eq:notroot}
 \disc_{T}(h(T)-u-\theta_{i}^{(j)}) \quad \text{for}\quad  1 \leq i \leq
 r,\quad
1 \leq j \leq d_i, \end{equation} then $f_i(h(T)-a)$ is irreducible
over $\F_q$ if and only if $\C$ coincides with the Frobenius
conjugacy class $(\widetilde{M}/\F_{q}(u), P_a)$.
\end{lem}
\begin{proof} Since $a$ is not a root of any of the polynomials
\eqref{eq:notroot}, $P_a$ is unramified in $\widetilde{M}$, where
$P_a$ denotes the prime of $\F_q(u)$ corresponding to the
$(u-a)$-adic valuation on $\F_q(u)$. Now fix $1 \leq i \leq r$.
Using Lemma \ref{lem:subst} and Kummer's Theorem (\cite[Theorem
3.3.7]{stichtenoth93}), we find
\begin{align*}
 f_i(h(T)-a) \text{ is irreducible over $\F_q$} &\Longleftrightarrow
h(T) - a - \theta_i^{(1)} \text{ is irreducible over
$\F_{q^{d_i}}$} \\
&\Longleftrightarrow P_a \text{ stays prime in $K_{i,1}$}.
\end{align*}
This last possibility can be recast in terms of the action of
Frobenius. Let $\sigma$ denote any element of the Frobenius
conjugacy class $(M_i/\F_{q}(u), P_a)$; then necessarily
\begin{equation}\label{eq:restriction}
 \sigma \text{ restricts to the $q$th power map on
$\F_{q^{d_i}}$}. \end{equation} Moreover, $P_a$ stays prime in
$K_{i,1}$ if and only if
\begin{equation}\label{eq:galunion}
 \Gal(M_i/\F_q(u)) = \dot\bigcup_{l=0}^{d_i n -1}\Gal(M_i/K_{i,1})
\sigma^{l}. \end{equation} We now investigate when
\eqref{eq:galunion} holds.

Write $K_{i,1} = \F_{q^{d_i}}(u)(\alpha)$, where $\alpha \in
S_{i,1}$. Now \eqref{eq:restriction} implies that under $\iota_2$
the element $\sigma$ is identified with $(\Frob, \sigma')$, where
$\sigma'$ is a permutation of $\cup_{j=1}^{d_i}{S_{i,j}}$. We claim
that \eqref{eq:galunion} holds if and only if $\sigma'$ is an
$nd_i$-cycle. Indeed, suppose that $\sigma$ (equivalently,
$\sigma'$) acts as an $nd_i$-cycle on $\cup_{j=1}^{d_i}{S_{i,j}}$;
then for any $\gamma \in \Gal(M_i/\F_q(u))$, there is a unique $0
\leq l < d_i n$ for which $\tau \sigma^{-l}$ fixes $\alpha$, and
this implies \eqref{eq:galunion}. Conversely, if \eqref{eq:galunion}
holds then $\sigma \not\in \Gal(M_i/K_{i,1})$, so that $\sigma$ (and
hence $\sigma'$) must move $\alpha$. Thus in the decomposition of
$\sigma'$ into disjoint cycles, $\alpha$ must occur in a nontrivial
cycle. If this cycle has length $l < nd_i$, then both $\sigma^l$ and
$\sigma^0$ belong to $\Gal(M/K_{i,1})$, and this contradicts that
\eqref{eq:galunion} is a disjoint union.

Let $\gamma$ denote an element of the conjugacy class of
$(\widetilde{M}/\F_{q}(u), P_a)$. Since $\gamma$ restricts down to
an element of the conjugacy class of $(M_i/\F_q(u), P_a)$, in order
for $P_a$ to stay prime in $M_i$ for every $i=1, 2, \dots, r$ it is
necessary and sufficient that $\gamma|_{M_i}$ satisfies both
\eqref{eq:restriction} and \eqref{eq:galunion} for every $1\leq i
\leq r$. By our work above and the commutativity of diagram
\eqref{eq:cd}, this condition on $\gamma$ holds if and only if
$\gamma$ (identified with its representation under $\iota_1$) has
the form $(\Frob, \sigma_1, \dots, \sigma_r)$, where each $\sigma_i$
is an $nd_i$-cycle on $\cup_{j=1}^{d_i}S_{i,j}$. It remains to prove
that the $\gamma $ in $\Gal(\widetilde{M}/\F_{q}(u))$ of this form
make up a single conjugacy class of size $n^{-r} n!^{d_1 + \dots +
d_r}$.

Suppose that $\gamma \in \Gal(\widetilde{M}/\F_{q}(u))$ has the
above form. The compatibility condition of Lemma \ref{lem:iota1}(iv)
implies that
\[ \sigma_i(S_{i,j}) \subset S_{i, j+1} \quad \text{for all $1\leq i \leq r$
and all $j$}. \] Now fix $1 \leq i \leq r$. Since $\sigma_i$ is an
$nd_i$-cycle on $\cup_{j=1}^{d_i}S_{i,j}$, it follows that
$\sigma_i$ has exactly $n$ representations in the form
\begin{multline*} (a_1~a_2~\dots~a_{n d_i}), \quad\text{where for each
$1 \leq k \leq d_i$}, \\ (a_k~a_{k+d_i}~\dots~a_{(n-1)k+d_i}) \text{
is an $n$-cycle of $\Sym(S_{i,k})$}. \end{multline*} Consequently,
there are exactly $n^{-1} n!^{d_i}$ possibilities for $\sigma_i$,
and so exactly
\[n!^{-r} n!^{d_1 + \dots + d_r}\]  possibilities for $\gamma$.
Moreover, this explicit description shows that the $\gamma$ of this
form make up a single conjugacy class of
$\Gal(\widetilde{M}/\F_q(u))$. To see this observe that
\[ \Gal(\widetilde{M}/\F_q(u)) \supset \Gal(\widetilde{M}/\F_{q^D}(u)) = \prod_{\substack{1 \leq i \leq r \\ 1 \leq j \leq d_i}}
\Sym(S_{i,j})
\] and that $\Sym(S_{i,j})$ acts transitively by conjugation on its own
$n$-cycles.
\end{proof}

To apply the Chebotarev density theorem we require an estimate for
the genus of $\widetilde{M}/\F_{q^D}$. This will be obtained as a
corollary of the next result, which appears as \cite[Theorem
III.10.3]{stichtenoth93}:

\begin{cast} Let $F/k$ be a function field with full constant field
$k$. Suppose we are given two subfields $F_1/k$ and $F_2/k$ of $F/k$
satisfying
\begin{enumerate}
\item $F = F_1 F_2$ is the compositum of $F_1$ and $F_2$,
\item $[F: F_i] = n_i$ and $F_i/k$ has genus $g_i$ for $i=1, 2$.
\end{enumerate}
Then the genus $g$ of $F/k$ obeys the bound
\[ g \leq n_1 g_1 + n_2 g_2 + (n_1 -1)(n_2-1). \]
\end{cast}

\begin{cor}\label{lem:genus} Let $f_1(T), \dots, f_r(T)$ be pairwise nonassociated
monic irreducible polynomials of respective degrees $d_1, \dots,
d_r$ over $\F_q$ and suppose that $h(T)$ is a polynomial of degree
$n \geq 2$ without constant term satisfying conditions
\eqref{eq:disccond} and \eqref{eq:rescond}. Then the genus of
$\widetilde{M}/\F_{q^D}$ is bounded above by
\[ (2(d_1 + \dots + d_r)-1) n!^{d_1+\dots+d_r-1} n^n. \]
\end{cor}
\begin{proof} We make repeated use of Castelnuovo's Inequality. Our
first application is an estimate for the genus of the function
fields $\widetilde{L}_{i,j}/\F_{q^D}$. For a fixed pair of $i$ and
$j$, let $\widetilde{K}^{(1)}, \dots, \widetilde{K}^{(n)}$ be the
complete list of conjugate fields of $\widetilde{K}_{i,j}$, so that
$\widetilde{L}_{i,j}$ is the compositum of $\widetilde{K}^{(1)},
\dots, \widetilde{K}^{(n)}$. For any $m \leq n$, Castelnuovo's
Inequality implies that (using $g_N$ to denote the genus of
$N/\F_{q^D}$)
\begin{multline*}
g_{\widetilde{K}^{(1)}  \cdots \widetilde{K}^{(m)}} \leq \\
[\widetilde{K}^{(1)} \cdots \widetilde{K}^{(m)}: \widetilde{K}^{(1)}
\cdots \widetilde{K}^{(m-1)}] g_{\widetilde{K}^{(1)} \cdots
\widetilde{K}^{(m-1)}}
 +
[\widetilde{K}^{(1)} \cdots \widetilde{K}^{(m)}: \widetilde{K}^{(m)}] g_{\widetilde{K}^{(m)}} +\\
\left([\widetilde{K}^{(1)} \cdots \widetilde{K}^{(m)}:
\widetilde{K}^{(1)} \cdots
\widetilde{K}^{(m-1)}]-1\right)\left([\widetilde{K}^{(1)} \cdots
\widetilde{K}^{(m)}: \widetilde{K}^{(m)}]-1\right).
\end{multline*}
Since each $\widetilde{K}^{(i)}$ is a rational function field
(obtainable by adjoining a single root of $h(T)-u-\theta_i^{(j)}$ to
$\F_{q^D}$), we have $g_{\widetilde{K}^{(m)}} = 0$ and so the second
summand on the right hand side vanishes. Estimating the size of the
field extensions appearing here trivially, we find
\[
g_{\widetilde{K}^{(1)} \cdots \widetilde{K}^{(m)}} \leq n
g_{\widetilde{K}^{(1)} \cdots \widetilde{K}^{(m-1)}} +
(n-1)(n^{m-1}-1) \leq n g_{\widetilde{K}^{(1)} \cdots
\widetilde{K}^{(m-1)}} + n^{m-1}.
\]
This relation implies inductively that
\[ g_{\widetilde{K}^{(1)} \cdots \widetilde{K}^{(m)}} \leq (m-1) n^{m-1} \]
and so taking $m=n$ yields \[ g_{\widetilde{L}_{i,j}} \leq (n-1)
n^{n-1} \leq n^n,
\] say. To continue we enumerate the $\widetilde{L}_{i,j}$ as $\widetilde{L}^{(1)}, \dots,
\widetilde{L}^{(d_1+\dots+d_r)}$, so that $\widetilde{M}$ is the
compositum of the $\widetilde{L}^{(i)}$ for $1 \leq i \leq d_1 +
\dots + d_r$. By Castelnuovo's Inequality, we have for any $k \leq
d_1 + \dots + d_r$ that
\begin{multline*}
 g_{\widetilde{L}^{(1)} \cdots \widetilde{L}^{(k)}} \leq [\widetilde{L}^{(1)} \cdots \widetilde{L}^{(k)}: \widetilde{L}^{(k)}] g_{\widetilde{L}^{(k)}} +
  [\widetilde{L}^{(1)} \cdots \widetilde{L}^{(k)}: \widetilde{L}^{(1)} \cdots \widetilde{L}^{(k-1)}] g_{\widetilde{L}^{(1)} \cdots \widetilde{L}^{(k-1)}} + \\
  ([\widetilde{L}^{(1)} \cdots \widetilde{L}^{(k)}: \widetilde{L}^{(k)}]-1)([\widetilde{L}^{(1)} \cdots \widetilde{L}^{(k)}: \widetilde{L}^{(1)} \cdots
  \widetilde{L}^{(k-1)}]-1);
\end{multline*}
thus \begin{align*}
 g_{\widetilde{L}^{(1)} \cdots \widetilde{L}^{(k)}} &\leq n!^{k-1} n^n + n! g_{\widetilde{L}^{(1)} \cdots \widetilde{L}^{(k-1)}} + (n!^{k-1}-1)(n!-1)
\\ &\leq n!^{k-1} n^n + n!  g_{\widetilde{L}^{(1)} \cdots \widetilde{L}^{(k-1)}} + n!^{k-1} n^n.\end{align*}
Another induction now shows
\[ g_{\widetilde{L}^{(1)} \cdots \widetilde{L}^{(k)}} \leq (2k-1) n!^{k-1} n^n; \]
taking $k=d_1 + d_2 + \dots + d_r$ gives the result.
\end{proof}

Finally we state the particular version of the Chebotarev density
theorem required in our application. This result is implicit in
Fried \& Jarden's discussion of the Chebotarev density theorem (see
\cite[Proposition 6.4.8]{fj05} and its proof, which incorporates
corrections from \cite[Appendix]{gj98}); a similar explicit estimate
has been given by Murty \& Scherk \cite{MS94}.

\begin{cheb} Suppose $M/\F_{q}(u)$ is a finite
Galois extension having full field of constants $\F_{q^D}$. Let $\C$
be a conjugacy class of $\Gal(M/\F_{q}(u))$ every element of which
restricts down to the $q$th power map on $\F_{q^D}$. Let
\[ \P:= \left\{\text{first degree primes $P$ of $\F_q(u)$
unramified in $M$}: \leg{M/\F_q(u)}{P} = \C\right\}. \] Then
\[ \left|\#\P - \frac{\#C}{[M:\F_{q^D}(u)]}q\right| \leq 2 \frac{\#C}{[M:\F_{q^D}(u)]}(g q^{1/2} + g + [M:\F_{q^D}(u)]), \]
where $g$ denotes the genus of $M/\F_{q^D}$.
\end{cheb}

\begin{proof}[Proof of Theorem \ref{thm:main}] Suppose that the polynomial
$h(T) = T^n + a_{n-1}T^{n-1} + \dots + a_1 T$ over $\F_q$ satisfies
both \eqref{eq:disccond} and \eqref{eq:rescond}. The number of $a
\in \F_q$ for which at least one of the polynomials
\eqref{eq:notroot} vanishes is bounded above by
\[ (n-1)(d_1 + \dots + d_r) \leq (n-1)B. \] For all other $a \in \F_q$ the
simultaneous irreducibility of the $f_i(h(T)-a)$ is equivalent to
$(\widetilde{M}/\F_q(u), P_a)$ coinciding with the conjugacy class
$\C$ appearing in Lemma \ref{lem:class}. Since $\C$ has size $n^{-r}
n!^{d_1+\dots + d_r}$ and $[\widetilde{M}:\F_{q^D}(u)] =
n!^{d_1+\dots+d_r}$, the explicit Chebotarev density theorem implies
there are at least
\[
\frac{q}{n^r} - \frac{2}{n^r}\left(g q^{1/2} + g+
n!^{d_1+\dots+d_r}\right)-(n-1) B\
\] values of $a \in \F_q$ for which all the polynomials $f_i(h(T)-a)$ are
irreducible, and at most \[(n-1) B + \frac{q}{n^r} +
\frac{2}{n^r}\left(g q^{1/2} + g+ n!^{d_1+\dots+d_r}\right)
\]
such values of $a$. (Here $g$ denotes the genus of
$\widetilde{M}/\F_{q^D}(u)$.)

We now replace $d_1+\dots+d_r$ by $B$ and sum over the possibilities
for $h$. Assume that
\[ q^{n-1} > 4n^2 \left(1 + \binom{B}{2}\right), \]
which holds if $q$ is sufficiently large in terms of $n$ and $B$.
(This inequality guarantees that there is some $h$ of degree $n$ for
which \eqref{eq:disccond} and \eqref{eq:rescond} both hold. Note
that this inequality can be assumed for the proof of Theorem
\ref{thm:main}, since for $q$ bounded in terms of $n$ and $B$ the
estimate of that theorem is trivial.) Then we find that the total
number of monic degree $n$ polynomials $\widetilde{h}(T)$ for which
all the $f_i(\widetilde{h}(T))$ are irreducible is bounded below by
\begin{equation}\label{eq:lower}
\left(q^{n-1} - 4n^2 \left(1 + \binom{B}{2}\right)\right)\left(\frac{q}{n^r} - \frac{2}{n^r}\left(g q^{1/2} + g+ n!^{B}\right) - (n-1)B\right)
\end{equation}
and bounded above by
\begin{multline*} 4n^2 q^{n-1} \left(1 + \binom{B}{2}\right) + \\
\left(q^{n-1} - 4n^2 q^{n-2}\left(1 +
\binom{B}{2}\right)\right)\left(\frac{q}{n^r} +\frac{2}{n^r}\left(g
q^{1/2} + g+ n!^{B}\right)+ (n-1)B \right).
\end{multline*}
Since $g$ is $O_{n,B}(1)$ by Corollary \ref{lem:genus}, both the
upper and lower bounds have the form $q^n/n^r + O_{n,B}(q^{n-1/2})$,
finishing the proof.
\end{proof}

\section{Proof of Theorem \ref{thm:arith}}\label{sec:arithm} We begin with some comments on the relation
between Theorem A and Theorem \ref{thm:arith}. For $q$ large in
terms of $r$ and $B$, Theorem A asserts the existence of infinitely
many irreducibility preserving substitutions $T \mapsto
T^{l^k}-\beta$ for some prime $l$ dividing $q-1$ and some $\beta \in
\F_q$. So we obtain irreducibility-preserving substitutions whose
degrees are exactly the powers of $l$. In the proof of Theorem A,
there is some control over the choice of $l$, and this could be used
to establish Theorem \ref{thm:arith} in a number of special cases.

In order to prove Theorem \ref{thm:arith} in full, we require two
additional ingredients:
\begin{enumerate}
\item the existence of a preliminary irreducibility-preserving substitution $T \mapsto h(T)$
of degree $d$, for an appropriate $d$ belonging to the progression
$a\bmod m$,
\item the existence of some $l$ prime to $m$ and some $\beta \in \F_q$ for which all the
substitutions $T\mapsto T^{l^k}-\beta$ preserve the irreducibility
of the polynomials $f_i(h(T))$, where $h(T)$ is as in (i).
\end{enumerate}
If we can establish (i) and (ii), then Theorem \ref{thm:arith}
follows immediately, since $h(T^{l^k}-\beta)$ has degree from the
progression $a\bmod m$ whenever $k$ is divisible by $\varphi(m)$.
The most difficult part of the proof is obtaining (i), which
requires Theorem \ref{thm:main}. By contrast, the techniques
necessary for the proof of (ii) are present already in
\cite{pollack06a}. However, the details here are slightly different;
this is because in proving Theorem \ref{thm:arith} we take $l$ as a
divisor of $q^d-1$ (with $d$ as in (i) above), while in the cited
paper $l$ is always chosen as a divisor of $q-1$.

We now give the specifics. Recall the following elementary result of
Bang \cite{bang86} (see \cite[Theorem 3]{roitman97} for a short
modern account):
\begin{bang} Let $a$ and $d$ be integers greater than $1$. Then
there is a prime $p$ for which $a$ has order $d$ modulo $p$ in all
except the following cases:
\begin{enumerate}
\item $d=2$, $a=2^s-1$, where $s\geq 2$,
\item $d=6$, $a=2$.
\end{enumerate}
\end{bang}

\begin{cor}\label{cor:bangcor} Let $m$ be a positive integer. Then every integer $d > \max\{2, \varphi(m)\}$ has the following property: if $q$ is any odd integer $\geq
3$, then $q^d-1$ has an odd prime divisor not dividing $m$.
\end{cor}
\begin{proof} Suppose $d >
\max\{2,\varphi(m)\}$. By Bang's theorem there is a prime $l$ for
which $q$ has order $d$ in $(\Z/l\Z)^{\times}$. Since $d > 1$, we
must have $l \neq 2$. Moreover, $l$ is necessarily prime to $m$: for
if $l$ divides $m$, then the order of $q$ in $(\Z/l\Z)^{\times}$ is
a divisor of $\varphi(l)$, hence also a divisor of $\varphi(m)$ and
so less than $d$, a contradiction. Hence $l$ is an odd prime divisor
of $q^d-1$ which is prime to $m$.
\end{proof}

The next lemma, due to Serret in the case of prime fields
\cite[Th\'{e}or\`{e}me I, p. 656]{serret86} and Dickson in the
general case (\cite[p. 382]{dickson97}; see also
\cite[\S34]{dickson58}), plays an essential role in the proofs of
both Theorems \ref{thm:arith} and \ref{thm:hall}. (For a modern
treatment see \cite[Theorem 3.3.5]{ln97}.) Recall that if $f(T)$ is
an irreducible polynomial over $\F_q$ not associated to $T$, then by
the \emph{order of $f$} we mean the order of any of its roots in the
multiplicative group of its splitting field (equivalently, the order
of $T$ in the unit group $(\F_q[T]/f)^{\times}$). Thus if $f$ has
degree $d$, then the order of $f$ is a divisor of $q^d-1$.

\begin{lem}[(Serret, Dickson)]\label{lem:dickson} Let $f$ be an irreducible
polynomial over $\F_q$ of degree $d$ and order $e$. Let $l$ be an
odd prime. Suppose that $f$ has a root $\alpha \in \F_{q^d}$ which
is not an $l$th power, or equivalently that
\begin{equation}\label{eq:hallcond} l \mid e \quad \text{but}
\quad l \nmid (q^{d}-1)/e.
\end{equation}
Then the substitution $T \mapsto T^{l^k}$ leaves $f$ irreducible for
every $k=1, 2, 3,\dots$.
\end{lem}

We also require the following estimate for character sums which
appears as \cite[Lemma 7]{pollack06a}:
\begin{lem}\label{lem:charest} Let $f_1(T), \dots, f_s(T)$ be pairwise nonassociated
irreducible polynomials over $\F_q$ with the degree of $f_1 \cdots
f_s$ bounded by $B$. Fix roots $\alpha_1, \dots, \alpha_s$ of $f_1,
\dots, f_s$, respectively, lying in an algebraic closure of $\F_q$.
Suppose that for each $i=1, 2, \dots, s$ we have a character
$\chi_i$ of $\F_{q}(\alpha_i)$ and that at least one of these
$\chi_i$ is nontrivial. Then
\begin{equation}\label{eq:wanbound} \left|\sum_{\beta \in
\F_q}\chi_1(\alpha_1+\beta) \cdots \chi_s(\alpha_s+\beta)\right|
\leq (B - 1)\sqrt{q}.
\end{equation}
\end{lem}

We can now establish the following variant of Theorem A:

\begin{lem}\label{lem:variant} Let $f_1(T), \dots, f_r(T)$ be pairwise nonassociated
irreducible polynomials over $\F_q$ with each $f_i$ of degree $>1$
and the degree of $f_1\cdots f_r$ bounded by $B$. Suppose that there
is a common odd prime $l$ dividing $q^{\deg{f_i}}-1$ for each
$i=1,2,\dots,r$. If
\[ q > (2^{r-1}B - 2^r + 1)^2, \]
then there is a $\beta \in \F_q$ for which all the polynomials
$f_1(T^{l^k}-\beta), \dots, f_r(T^{l^k}-\beta)$ are irreducible for
each $k=1,2,3, \dots$.
\end{lem}
\begin{proof} Fix roots $\alpha_1, \dots, \alpha_r$ of
$f_1(T), \dots, f_r(T)$, respectively. By Lemma \ref{lem:dickson},
it suffices to produce an element $\beta \in \F_q$ with the property
that $\alpha_i + \beta$ is an $l$th power nonresidue in
$\F_{q}(\alpha_i)$ for every $i=1, 2, \dots, r$. Since $l$ divides
$q^{\deg{f_i}}-1$ for each $i$, there are characters $\chi_i$ of
order $l$ on each of the fields $\F_{q}(\alpha_i)$. If for every
choice of $\beta$, there is an $i \in \{1, 2, \dots, r\}$ for which
$\alpha_i+\beta$ is an $l$th power in $\F_{q}(\alpha_i)$, then the
sum \[ \sum_{\beta \in \F_q}(1-\chi_1(\alpha_1+\beta))
(1-\chi_2(\alpha_2+\beta)) \cdots (1-\chi_r(\alpha_r+\beta)) \]
vanishes. (Note that it is impossible for any of the arguments
$\alpha_i+\beta$ inside a character to vanish, since each $\alpha_i$
belongs to a nontrivial extension of $\F_q$.) But by Lemma
\ref{lem:charest}, the absolute value of this sum is bounded below
by
\begin{multline*}
 q - \sum_{\substack{\I \subset \{1, 2, \dots, r\}\\ \I \neq \emptyset}}\left(-1 + \sum_{i \in
 \I}\deg{f_i(T)}\right)\sqrt{q}= \\ q + (2^{r}-1)\sqrt{q} - \sum_{i=1}^{r}\deg{f_i}
 \left(\sum_{\substack{\I \subset \{1, 2, \dots, r\} \\ i
\in I}}1\right)\sqrt{q} \geq \\ q + (2^r-1)\sqrt{q} - 2^{r-1} B
\sqrt{q},
\end{multline*}
and this is positive for $q$ as in the hypothesis of the lemma.
\end{proof}

\begin{proof}[Proof of Theorem \ref{thm:arith}] Suppose $f_1, \dots,
f_r$ are irreducible polynomials over $\F_q$, where $\F_q$ is a
finite field with characteristic $p$ coprime to $2\gcd(a,m)$. Let
$d$ be the smallest integer exceeding $\max\{2, \varphi(m)\}$
relatively prime to $p$ and satisfying $d\equiv a \pmod{m}$. Since
$p$ is prime to $\gcd(a,m)$, it follows that $p$ divides at most one
of any two consecutive terms from the progression $a\bmod{m}$, so
that $d \leq 3m$. In particular $d$ is bounded solely in terms of
$m$. So by Theorem \ref{thm:main}, as long as $q$ is sufficiently
large (depending just on $B$ and $m$), there is a polynomial $h$ of
degree $d$ for which all of $f_1(h(T)), \dots, f_r(h(T))$ are
irreducible over $\F_q$. Using Corollary \ref{cor:bangcor}, choose a
prime $l$ dividing $q^d-1$ which is relatively prime to $m$. Then
$l$ also divides $q^{\deg{f_i(h(T))}}-1$ for each $i=1,2,\dots,r$.
According to Lemma \ref{lem:variant} (applied to the polynomials
$f_1(h(T)), \cdots, f_r(h(T))$), if
\[ q> (2^{r-1}dB - 2^r + 1)^2, \] then there is some $\beta \in
\F_q$ with the property that the polynomials $f_i(h(T^{l^k}-\beta))$
are all irreducible over $\F_q$ for $k=1, 2, 3, \dots$. Since
\[ \deg{h(T^{l^k}-\beta)} = d l^k \equiv a l^k \equiv a
\pmod{m} \] whenever $k$ is a multiple of $\varphi(m)$, the proof of
Theorem \ref{thm:arith} is complete.\end{proof}


\section{Application to a Question of Hall}
We prove Theorem \ref{thm:hall} in two parts:
\subsection{Part I: Infinitely Many Twin Prime Pairs of Odd Degree}
In the case when $q-1$ has an odd prime divisor the twin prime pairs
constructed by Hall \cite{hall06} already have odd degree, so we may
suppose that $q-1$ is a power of $2$. Now recall that if $q$ is an
odd prime power for which $q-1$ is a power of $2$, then either $q=9$
or $q$ is a Fermat prime (\cite[p. 374, Exercise 1]{sierpinski88}).

Theorem \ref{thm:arith} guarantees the existence of a twin prime
pair $f, f+1$ of degree $\equiv 1 \pmod{2}$ over all sufficiently
large finite fields $\F_q$ with $q$ odd. The next lemma is an
explicit version of a slightly weaker result:

\begin{lem}\label{lem:hall} Suppose $q > 10^6$ is a prime power
coprime to $6$. Then there are infinitely many twin prime pairs
$f,f+1$ over $\F_q$ for which $\deg{f} = \deg{(f+1)}$ is odd.
\end{lem}

It is worth remarking that no Fermat primes $> 10^6$ are known, and
it is plausible that none exist.

\begin{proof} By Theorem
\ref{thm:main}, if $q$ is large enough and prime to $6$, then we may
choose a monic prime pair $f, f+1$ of degree $3$ over $\F_q$. In
fact, referring to the lower bound \eqref{eq:lower} (with $r=2$,
$B=2$ and $n=3$), we see that such pairs exist as long as $q$
satisfies the inequalities
\begin{equation}\label{eq:explicit}
 q^2 > 8\cdot 3^2 \quad \text{and}\quad \frac{q}{9} - \frac{2}{9}(g q^{1/2} + g + 6^2) - 2\cdot 2 >
 0,
\end{equation}
where $g$ is the genus of an appropriate function field. The left
hand inequality is satisfied already for $q \geq 9$. By Corollary
\ref{lem:genus}, we have
\[ g \leq (2\cdot 2-1) 6^{2-1} 3^3 = 486; \]
and so the right hand inequality \eqref{eq:explicit} holds as soon
as
\[ \frac{1}{9} q - 108\sqrt{q} - 120 > 0, \]
which is valid for $q \geq 946943$, so certainly for $q > 10^6$. To
complete the proof, choose an odd prime divisor $l$ of $q^3-1$
(e.g., any prime divisor of $q^2+q+1$) and apply Lemma
\ref{lem:variant} to the pair $f, f+1$ (taking $B=6$ and $r=2$). We
obtain that for $q> 81$, there is some $\beta \in \F_q$ for which
both $f(T^{l^k}-\beta)$ and $f(T^{l^k}-\beta)+1$ are simultaneously
irreducible for $k = 1, 2, 3, \dots$. This is an infinite family of
twin prime pairs of odd degree.
\end{proof}
\begin{table} \caption{\label{tbl:smalltwins} For each odd prime power $q = 2^N+1$ not exceeding $10^6$,
we exhibit a monic prime polynomial $f$ of odd degree $d$ over
$\F_q$ for which $f+1$ is also prime, together with the
factorization of $q^d-1$, the factorizations of the order of $f$ and
$f+1$, and an odd prime $l$ for which Lemma \ref{lem:dickson} can be
applied to both $f$ and $f+1$. We write $P_9$ for the $9$-digit
prime $116085511$.} \small \centering \begin{tabular}{r|r|r|r|r|r}
$q$ & $f$ &\textbf{$q^d-1$} & \textbf{order of $f$} & \textbf{order
of $f+1$} & \textbf{$l$}\\\hline\hline\rule{0pt}{9pt} $3$ &
$T^3-T+2$ & $2\cdot 13$ & $13$ & $26$ & $13$\\
$9$ &
$T^3-T+2$ & $2^3 \cdot 7 \cdot 13$ & $13$ & $26$ & $13$\\
$5$ & $T^3+3T + 2$ & $2^2 \cdot 31$ & $2^2\cdot 31$ & $2^2\cdot 31$
&
$31$ \\
$17$ & $T^3 + T + 8$ & $2^4\cdot 307$ & $2^2\cdot 307$ & $2^2\cdot
307$ & $307$ \\
$257$ & $T^3+T+15$ & $2^8 \cdot 61 \cdot 1087$ & $2^5\cdot 61 \cdot
1087$ & $2^2\cdot 61 \cdot 1087$ & 61 \\ 65537 & $T^3+T+18$ &
$2^{16}\cdot 37\cdot P_9$ & $2^{15}\cdot 37\cdot P_9$ & $2^{15}\cdot
37\cdot P_9$ & 37
\end{tabular}
\normalsize
\end{table}

To finish off this half of Theorem \ref{thm:hall}, it remains to
consider the cases when $q=9$ or when $q$ is a Fermat prime $<
10^{6}$. These small finite fields are treated by hand. For each
such $q$, Table \ref{tbl:smalltwins} exhibits the first member $f$
of a monic twin prime pair $f, f+1$ of odd degree together with all
the information necessary to verify that Lemma \ref{lem:dickson} can
be applied to both $f$ and $f+1$ with the specified odd prime $l$.

\subsection{Part II: Infinitely Many Twin Prime Pairs of
Even Degree} We first argue that for $q \geq 4$, there is always a
monic, quadratic twin prime pair $f, f+1$ over $\F_q$. In the proof
of this result it is convenient to consider odd and even $q$
separately.

\begin{lem}\label{lem:quadratic1} Let $\F_q$ be a finite field of odd characteristic with $q\geq 5$. Then there is
a pair $f, f+1$ of monic irreducible quadratic polynomials over
$\F_q$.
\end{lem}

Lemma \ref{lem:quadratic1} could be established by the methods of
Theorem \ref{thm:main}, in analogy with the proof of Lemma
\ref{lem:hall} in Part I. However, the direct approach below leads
to better bounds.

\begin{proof} It suffices to show that there is some pair of consecutive quadratic
nonresidues in $\F_q$. Letting $\chi$ denote the quadratic character
on $\F_q$, the number of such pairs is $\frac{1}{4}$ of the sum
$\sum(1-\chi(\alpha))(1-\chi(\alpha+1))$, the sum being taken over
$\alpha \neq 0, -1$ from $\F_q$. Now a straightforward calculation
using the evaluation $\sum_{\alpha \in \F_q}
\chi(\alpha)\chi(\alpha+1) = -1$ (cf. \cite[Theorem 2.1.2]{bew98})
results in a count of
\[ \frac{1}{4}\left(q-3 +\chi(1) + \chi(-1)\right) = \frac{1}{4}
\left(q-2 + \chi(-1)\right)\] such pairs, which is positive for $q >
3$.
\end{proof}
\begin{lem}\label{lem:quadratic2} Let $\F_q$ be a finite field of characteristic $2$ with $q\geq 4$.
Then there is a pair $f, f+1$ of monic quadratic polynomials both of
which are irreducible over $\F_q$.
\end{lem}
\begin{proof} For any fixed $\gamma \in \F_q$, the map $\phi\colon
\F_q\mapsto\F_q$ defined by $\phi(\beta) := \beta^2+\gamma\beta$ is
an endomorphism of the underlying additive group of $\F_q$. We
choose $\gamma$ so that $\gamma \neq 0$ and the image of $\phi$
contains $1$ (and so contains all of $\F_2$). This is possible as
soon as $\F_q$ is a nontrivial extension of $\F_2$; merely choose
any $\beta \in \F_q\setminus \F_2$ and define $\gamma$ so that
$\beta^2+\gamma\beta = 1$.

We claim that with this choice of $\gamma$, there is a pair $f, f+1$
of irreducibles where $f$ has the form $T^2+\gamma T + \delta$. A
polynomial of this form is irreducible if and only if $\delta$ is
not in the image of $\phi$. But by our choice of $\gamma$, the
element $\delta$ is missing from the image of $\phi$ if and only if
the same is true for $\delta+1$. So the lemma follows provided that
$\phi$ is not onto. Since $\phi$ is a map from $\F_q$ to itself, if
$\phi$ were onto it would also be injective. But $\phi(\gamma) =
\phi(0) = 0$, and the lemma is proved.\end{proof}

\begin{lem} Let $\F_q$ be a finite field with $q > 25$. Then there
are infinitely many twin prime pairs $f, f+1$ of even degree over
$\F_q$.
\end{lem}
\begin{proof} Lemmas \ref{lem:quadratic1} and
\ref{lem:quadratic2} show that for $q \geq 4$ there is a monic twin
prime pair $f,f+1$ of degree $2$ over $\F_q$. Since $q > 3$, it is
impossible for both $q-1$ and $q+1$ to be powers of $2$, and so
there must be an odd prime divisor $l$ of $q^2-1$. Lemma
\ref{lem:variant} (with $r=2$ and $B=4$) implies that for $q> 25$,
there is some $\beta \in \F_q$ for which both $f(T^{l^k}-\beta)$ and
$f(T^{l^k}-\beta)+1$ are simultaneously irreducible for $k= 1, 2, 3,
\dots$. Since these twin prime pairs have even degree, the lemma
follows.
\end{proof}

To complete the proof of Theorem \ref{thm:hall} it suffices to
consider those finite fields with at most $25$ elements, and these
are treated in Table \ref{tbl:smalltwins2}.

\begin{table} \caption{\label{tbl:smalltwins2} For each prime power $2 < q \leq 25$
we exhibit a monic prime polynomial $f$ of even degree $d$ over
$\F_q$ for which $f+1$ is also prime, together with the
factorization of $q^d-1$, the factorizations of the order of $f$ and
$f+1$, and a prime $l$ for which Lemma \ref{lem:dickson} can be
applied to both $f$ and $f+1$.} \small \centering
\begin{tabular}{r|r|r|r|r|r} $q$ & $f$ &\textbf{$q^d-1$} &
\textbf{order of $f$} & \textbf{order of $f+1$} &
\textbf{$l$}\\\hline\hline\rule{0pt}{9pt} $3$ &
$T^6+T^5+2T^3+2T^2+1$ & $2^3\cdot 7 \cdot 13$ & $2^2 \cdot 7\cdot 13$ & $2^3\cdot 7 \cdot 13$ & $7$\\
$4$ & $T^2+T + \alpha$ & $3\cdot 5$ & $3\cdot 5$ & $3 \cdot 5$ & $3$\\
$5$ & $T^2+T+1$ & $2^3 \cdot 3$ & $3$ & $2^3\cdot 3$ &$3$ \\
$7$ & $T^2 + T + 3$ & $2^4\cdot 3$ & $2^4\cdot 3$ & $2^3
\cdot 3$ & $3$ \\
$8$ & $T^2+(\beta+1)T+\beta^2+\beta$ & $3^2\cdot 7$ & $3^2\cdot 7$ & $3^2\cdot 7$ & $7$ \\
$9$ & $T^2+(\gamma+1)T + \gamma+1$ & $2^4 \cdot 5$ & $2^4 \cdot 5$ & $2^4\cdot 5$ & $5$ \\
$11$ & $T^2 + 3$ & $2^3\cdot 3\cdot 5$ & $2^2\cdot 5$ & $2^2\cdot 5$ & $5$ \\
$13$ & $T^2 + 6$ & $2^3\cdot 3 \cdot 7$ & $2^3\cdot 3$ & $2^3\cdot 3$ & $3$\\
$16$ & $T^2+ (\delta^2+\delta)T + \delta$ & $3 \cdot 5 \cdot 17$ & $3\cdot 5  \cdot 17$ & $3\cdot 5 \cdot 17$ & $3$\\
$17$ & $T^2+T+2$ & $2^5\cdot 3^2$ & $2^4\cdot 3^2$ & $2^5 \cdot 3^2$ & $3$\\
$19$ & $T^2+4$ & $2^3\cdot 3^2 \cdot 5$ & $2^2\cdot 3^2$ & $2^2\cdot 3^2$ & $3$\\
$23$ & $T^2+2$ & $2^4 \cdot 3 \cdot 11$ & $2^2\cdot 11$ & $2^2\cdot 11$ & $11$\\
$25$ & $T^2 + 4\epsilon T + 4\epsilon + 2$ & $2^4\cdot 3\cdot 13$ &
$3\cdot 13$ & $2^2\cdot 3\cdot 13$ & $3$\\
\end{tabular}
\\\vskip 2pt
Here $\alpha^2+\alpha+1=0$, $\beta^3+\beta+1 =0$, $\gamma^2+1=0$,
$\delta^4 + \delta + 1 =0$, and $\epsilon^2 + 2=0$.
\end{table}

\begin{acknowledgements}
Mitsuo Kobayashi and Tom Shemanske read over preliminary versions of
this paper and offered suggestions which proved indispensable in
coaxing it into its present form. Particular thanks are due to my
advisor, Carl Pomerance, both for his helpful advice and
encouragement on the paper at every stage of its development and for
his initial suggestion that I attempt to prove Theorem
\ref{thm:main} in its present generality.\end{acknowledgements}
\bibliographystyle {amsalpha}
\bibliography{twins}
\end{document}

\documentclass[12pt]{article}
%\usepackage[math]{kurier}
\usepackage{amsmath,amsthm,amssymb}

%\usepackage{graphicx}
\usepackage{bbold}
\usepackage{url}
\usepackage{soul}
\usepackage{geometry}
\def\C{\mathbf{C}}
\def\Z{\mathbf{Z}}
\def\Q{\mathbf{Q}}
\def\R{\mathbf{R}}
\def\Pp{\mathcal{P}}
\def\lcm{\mathrm{lcm}}
\def\proj{\mathrm{proj}}
\def\ord{\mathrm{ord}}
\def\bz{\mathbb{1}}
\def\1{\mathbb{1}}
\def\Arg{\mathrm{Arg}}
\def\Log{\mathop{\mathrm{Log}}}
\def\Arctan{\mathop{\mathrm{Arctan}}}

\renewcommand\Re{\mathrm{Re}}
\renewcommand\Im{\mathrm{Im}}
\newcommand{\leg}[2]{\genfrac{(}{)}{}{}{#1}{#2}}
\newcommand{\vc}[1]{\mathbf{#1}}
\theoremstyle{plain}
\newtheorem*{lem}{Lemma}
\newtheorem*{thm}{Theorem}

\theoremstyle{remark}
\newtheorem*{rmk}{Remark}
\geometry{left=1in, right=1in, top=.75in, bottom=.75in}
\makeatletter
\let\@@pmod\pmod
\DeclareRobustCommand{\pmod}{\@ifstar\@pmods\@@pmod}
\def\@pmods#1{\mkern4mu({\operator@font mod}\mkern 6mu#1)}
\makeatother
\begin{document}
\thispagestyle{empty} \begin{center} {\textbf{MATH 3200 --
Learning objectives to meet for Exam \#1}}
\end{center}
\noindent The exam will cover Chapters 1--3 of the course notes, discussing mathematical statements, proof methods, and induction, respectively. While you \emph{are} responsible for reading the notes, the exam will not ask about definitions or concepts not covered in class or on homework.

\subsection*{What to be able to state}

\subsubsection*{Basic definitions}
Be able to give concise, complete, and precise definitions of each of the following terms.
\begin{itemize}
\item statement
\item compound statement
\item logically equivalent (compound) statements
\item implication
\item converse, inverse, contrapositive (of an implication)
\item direct proof
\item proof by contrapositive
\item proof by contradiction
\item even and odd integers, parity
\item ``$a$ divides $b$'', where $a$ and $b$ are integers
\item induction 
\item strong (or complete) induction
\end{itemize}

\noindent To give you some idea of what ``complete'' means here: If you are asked for the definition of \textbf{even}, it is \emph{not} sufficient for you to write ``$2k$'' or ``$n=2k$.'' A satisfactory answer would be ``An integer $n$ is called \emph{even} when $n=2k$ for some integer $k$.''

\subsubsection*{What to be able to do}
You may be called on to perform any or all of the following tasks.
\begin{itemize}
\item Recognize examples and non-examples of mathematical statements
\item Determine whether compound statements are true or false, given the truth values of the component statements
\item Write out truth tables for compound mathematical statements
\item Use truth tables for two compound statements to demonstrate that they are (or are not) logically equivalent
\item Be able to negate mathematical statements, including compound statements involving quantifiers and connectives. Here the kind of negation you are expected to provide is one moving the ``not'' as far into the sentence as possible. You should be able to do this whether the initial statement is described in words or in symbols.
\item Determine whether simple mathematical statements are true or false (e.g., $\forall x \in \mathbb{R}, \forall y \in\mathbb{R},~x - y < x+y$).
\item Formulate proofs of statements involving parity, divisibility, and/or inequalities. (For inequality proofs, you will be provided with the list of rules.) Note that you may be required to determine yourself which proof method is appropriate. Your proofs should be logically correct and written coherently in complete sentences; \ul{you can expect me to assess both aspects when assigning your 
grades}.
\item Prove mathematical statements by induction and/or strong induction.

\end{itemize}

\subsection*{What to expect on the exam}
You can expect $5$ (possibly multi-part) questions on the exam. These will include:
\begin{itemize}
\item at least one problem asking you to fill in truth tables
\item at least one problem assessing your negation skills
\item at least one induction problem (possibly testing strong induction)
\end{itemize}


\newpage

\subsection*{Sample problems}

\begin{enumerate}


\item Write the negation of each of the following statements. Do \emph{not} justify your answers on this problem. For (a) and (b) only, determine whether the ORIGINAL statement is \texttt{TRUE} or \texttt{FALSE}.
\begin{enumerate}
\item Every real number that is greater than 100 is also greater than 1000.
\item There is a real number that is both greater than zero and less than zero.
\item For all triples of integers $x$, $y$, and $z$, if $x^3+y^3+z^3=0$, then $xyz=0$. 
\end{enumerate}

\item 
\begin{enumerate}
\item Write out the truth tables for $(P\Rightarrow Q) \Rightarrow R$ and $P\Rightarrow (Q\Rightarrow R)$. 
\item Are $(P\Rightarrow Q) \Rightarrow R$ and $P\Rightarrow (Q\Rightarrow R)$ logically equivalent?
\end{enumerate}

\item 
\begin{enumerate}
\item What does it mean to say $n$ is an \textbf{even} integer? \textbf{odd} integer?
\item Show that if $x$ and $y$ are integers and $x+y$ is even, then $x$ and $y$ are both even or both odd.
\end{enumerate}

\item Suppose $a$, $b$, and $c$ are integers.
\begin{enumerate}
\item What does it mean to say that $a$ divides $b$?
\item Show that if $a$ divides $b$ and $a$ divides $c$, then $a$ divides $2024b + 2025c$.
\end{enumerate}



\item 
\begin{enumerate}
\item Carefully state the \textbf{Axiom of Mathematical Induction}.
\item Prove that $1 + 4 + 7 + \dots + (3n- 2) = n(3n - 1)/2$ for every natural number $n$.
\end{enumerate}




\end{enumerate}
\end{document}


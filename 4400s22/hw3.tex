\documentclass[12pt]{article}
\usepackage{amsmath}
\usepackage{amsfonts}
\usepackage{amssymb}
\usepackage{graphicx}
\usepackage{geometry}
\usepackage{amsthm}
\usepackage{url}
\def\Z{\mathbf{Z}}
\def\Q{\mathbf{Q}}
\def\C{\mathbf{C}}
\def\R{\mathbf{R}}
\def\lcm{\mathrm{lcm}}
\def\ord{\mathrm{ord}}
\def\proj{\mathrm{proj}}
\newcommand{\leg}[2]{\genfrac{(}{)}{}{}{#1}{#2}}

\renewcommand\subset\subseteq
\newcommand{\vc}[1]{\mathbf{#1}}
\geometry{left=1in, right=1in, top=.75in, bottom=.75in}
\begin{document}
\thispagestyle{empty} \begin{center} {\textbf{MATH 4400/6400 --
Homework \#3}\\ posted Feb. 14; due Feb. 23, by midnight}
\end{center}

{\scriptsize \begin{quote} It is impossible to be a mathematician without being a poet in soul.\\
    -- Sofia Kovalevskaya\end{quote}}


\vskip 10pt \noindent\textbf{Directions}. Give complete solutions, providing full justifications when appropriate. Your assignment must be stapled if it goes on beyond one page.

\vskip 10pt \noindent\textbf{MATH 4400 problems}
\begin{enumerate}
\item Let $p$ be an odd prime, and let $a$ be an integer not divisible by $p$. Prove that $\sqrt{a}$ exists in $\Z_p$ if and only if $a^{(p-1)/2} \equiv 1\pmod{p}$.

{\scriptsize \emph{Hint}. Revisit the argument for when $\sqrt{-1}$ exists in $\Z_p$. That is, pair nonzero elements of $\Z_p$ that multiply to $a$.}

\item Prove the Division Algorithm in $\Z[\sqrt{2}]$: For every $\alpha, \beta \in \Z[\sqrt{2}]$ with $\beta \ne 0$, there are $\gamma, \rho \in \Z[\sqrt{2}]$ with $\alpha = \beta\gamma+\rho$ and $|N\rho| < |N\beta|$. Then do the same for $\Z[\sqrt{3}]$.

\item Show  that the equation $2\cdot 2 = (\sqrt{5}+1)(\sqrt{5}-1)$ exhibits a genuine failure of unique factorization in $\mathbb{Z}[\sqrt{5}]$. That is, show that all the factors involved are prime in $\Z[\sqrt{5}]$ and that the two factorizations cannot be made to agree with each other by reordering and introduction of unit factors.

\item Let $d \in \Z^{+}$ with $d\ne \square$. Show that for each positive integer $N$, there are integers $a, b$ with $1 \le b \le N$ and $|a+b\sqrt{d}| < 1/N$.

{\scriptsize \emph{Hint}. Put each of the the fractional parts $\{0\sqrt{d}\}, \dots, \{N\sqrt{d}\}$ into one of the $N$ intervals  $[0,1/N)$, $[1/N, 2/N)$, \dots, $[(N-1)/N,1)$. Then apply the Pigeonhole principle.}

\item Let $d \in \Z^{+}$ with $d\ne \square$. Suppose $a, b \in \Q$, and let $\eta = a+b\sqrt{d}$.
\begin{enumerate}
\item Expand $(x-\eta)(x-\tilde{\eta})$ in the form $x^2-Ax-B$, expressing $A$ and $B$ in terms of $a$ and $b$.
\item Show that if $x_n, y_n$ are defined by $x_n + y_n\sqrt{d} = (a+b\sqrt{d})^n$, then for every positive integer $n$,
\[ x_{n+1} = Ax_n + Bx_{n-1}, \quad y_{n+1} = Ay_n + By_{n-1}, \]
where $A$ and $B$ are the numbers you found in part (a).
\end{enumerate}

\item Let $d\in\Z^{+}$ with $d\ne \square$.
\begin{enumerate}
\item Suppose (as we will show in class is always the case) that there is a unit $>1$ in $\Z[\sqrt{d}]$. Prove there is a smallest unit $>1$ in $\Z[\sqrt{d}]$. 
\item Let $\epsilon$ be the smallest unit $>1$ in $\Z[\sqrt{d}]$. Show that the collection of units $> 1$ consists precisely of the elements $\epsilon^n$, for $n\in \Z^{+}$.
\item Show that the collection of all units in $\Z[\sqrt{d}]$ consists precisely of the elements $\pm \epsilon^n$, where now $n$ ranges over all of $\Z$.
\item With $\epsilon$ as in (b), show that if $N(\epsilon)=1$, then every unit in $\Z[\sqrt{d}]$ has norm $1$.
\end{enumerate}

\item Find the smallest unit $>1$ \underline{of norm $1$} in $\Z[\sqrt{99}]$. Then do the same for $\Z[\sqrt{101}]$. Justify your answers.

\item A number is called \emph{pentagonal} if it has the form $\frac{1}{2}n(3n-1)$ for some integer $n$.\footnote{If you are curious about the name, draw some dot diagrams of nested pentagons and count the number of dots at each stage. If you get stuck, check out \url{https://en.wikipedia.org/wiki/Pentagonal_number}.}

Consider the problem of finding all square pentagonal numbers, i.e., all positive integers $n$ and $m$ with $m^2 = \frac{1}{2}n(3n-1)$. The smallest $n$ which gives rise to a solution is $n=1$, corresponding to $m=1$. The second smallest $n$ is $n=81$, corresponding to $m=99$. Find, with proof, the third smallest $n$.

% \item Evaluate all of the following Legendre symbols, explaining your answer: $\leg{-1}{11}, \leg{2}{7}$, $\leg{-1000}{3}$, $\leg{1000}{5}$.

% \item Is there a square number of the form $5k-1$ (with $k$ a positive integer)? What about $55k-1$? What about $555k-1$?

\end{enumerate}

\vskip 10pt \noindent\textbf{MATH 6400 problems}
\begin{enumerate}
\item[G1.]Consider the sequence of primes $2, 3, 7, 43, 139, \dots$ defined by the following procedure. Let $q_1 = 2$, and assuming $q_j$ has been defined for $1\le j\le k$, let $q_{k+1}$ be the largest
prime divisor of $1 + q_1 \cdots q_k$.  Prove
that the prime 5 does not appear in the sequence $\{q_i\}_{i=1}^{\infty}$.

\item[G2.] Show that if $p\equiv 1\pmod{4}$ is prime, then $\Z_p$ contains a fourth root of $-4$.
\end{enumerate}


\end{document}

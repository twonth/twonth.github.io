\documentclass[12pt]{article}
\usepackage{amsmath}
\usepackage{amsfonts}
\usepackage{amssymb}
\usepackage{graphicx}
\usepackage{geometry}
\usepackage{amsthm}
\usepackage{url}
\def\Z{\mathbf{Z}}
\def\Q{\mathbf{Q}}
\def\C{\mathbf{C}}
\def\R{\mathbf{R}}
\def\lcm{\mathrm{lcm}}
\def\ord{\mathrm{ord}}
\def\proj{\mathrm{proj}}
\newcommand{\leg}[2]{\genfrac{(}{)}{}{}{#1}{#2}} 

\renewcommand\subset\subseteq
\newcommand{\vc}[1]{\mathbf{#1}}
\geometry{left=1in, right=1in, top=.75in, bottom=.75in}
\begin{document}
\thispagestyle{empty} \begin{center} {\textbf{MATH 4400/6400 --
Homework \#4}\\ posted \today; due March 25, by midnight}
\end{center}

{\scriptsize \begin{quote} Number theorists are like lotus-eaters --- having once tasted of this food they can never give it up.\\
    -- Leopold Kronecker\end{quote}}

 
\vskip 10pt \noindent\textbf{Directions}. Give complete solutions, providing full justifications when appropriate. Your assignment must be stapled if it goes on beyond one page.

\vskip 10pt \noindent\textbf{MATH 4400 problems}
\begin{enumerate}
\item 
\begin{enumerate}
\item Show that if $P,Q$ are odd integers, then $\frac{P^2-1}{8} + \frac{Q^2-1}{8} \equiv \frac{(PQ)^2-1}{8}  \pmod{2}$.
\item Prove (as claimed in class) that $\leg{2}{P} = (-1)^{(P^2-1)/8}$ for every odd positive $P\in \Z$.
\end{enumerate}

\item \begin{enumerate}
    \item Find $\leg{82}{365}$. Given that $365$ is not prime, what --- if anything --- can you conclude (without further calculation) from this about whether $82$ is a square mod $365$?
    \item Find $\leg{82}{367}$. Noting that $367$ is prime, what --- if anything --- can you conclude from this (without further calculation) about whether $82$ is a square mod $367$?
    \end{enumerate}

\item Let $a$ be a nonzero integer. Let $M = 4|a|$. Show that if $P,P'$ are odd positive integers with $P\equiv P'\pmod{M$}, then $\leg{a}{P} = \leg{a}{P'}$. (This says that the value of the symbol $\leg{a}{\cdot}$ depends only the ``denominator'' modulo $M$.)

{\scriptsize \emph{Hint.} Write $a=(\pm 1) \cdot 2^k \cdot b$ where $b$ is an odd positive integer. Show that $\leg{\pm 1}{P} = \leg{\pm 1}{P'}$, $\leg{2^k}{P} = \leg{2^k}{P'}$, and $\leg{b}{P} = \leg{b}{P'}$.}

\item Let $N$ be a positive integer. Prove that $k(N+1-k) \ge N$ for each integer $k=1,2,3,\dots,N$. Deduce that $(N!)^2 \ge N^N$.

\item Use calculus to show that $\frac{x}{\log{x}} > \sqrt{x}$ for every real number $x>1$. 
    
{\scriptsize \emph{Hint.} What does the graph of $\frac{x/\log{x}}{\sqrt{x}}$ look like? Remember that for us, $\log{x}$ means $\ln{x}$, the log base $\mathrm{e}$.}

\item Prove that for all real numbers $\alpha$ and $\beta$,
\[ \lfloor \alpha+\beta\rfloor - \lfloor \alpha\rfloor - \lfloor \beta\rfloor = 0\text{ or } 1.\]

    
\item Recall that $\ord_p(m)$ denotes the exponent on the largest power of the prime $p$ dividing the positive integer $m$. In class, we will show that if $p$ is a prime, and $n$ is a positive integer, then
\[ \ord_p(n!) = \sum_{k\ge 1} \lfloor n/p^k\rfloor. \]

Using this formula, determine the number of zeros at the end of $2021!$ (written in decimal, as usual).

\item Use Exercise 6 to prove that if $p$ is prime and $n$ is a positive integer, then
\[ \ord_p(n!) \ge \ord_p(k! (n-k)!)\quad\text{for all integers $0\le k \le n$}. \]

{\scriptsize [It follows that $k! (n-k)!$ divides $n!$. This gives another proof that the binomial coefficients $\binom{n}{k}$ are integers!]}

\item Define $\mathrm{Li}(x) = \int_{2}^{x} \frac{dt}{\log{t}}$; this function is called the \textsf{logarithmic integral}. Compute
\[ \lim_{x\to\infty} \frac{\mathrm{Li}(x)}{x/\log{x}}. \]

\end{enumerate}
 
\vskip 10pt \noindent\textbf{MATH 6400 problems}
\begin{enumerate}
\item[G1.] Let $p$ be a prime number. Show that if $p$ divides a number of the form $x^4 - x^2 + 1$, where
$x \in \Z$, then $p\equiv 1\pmod{12}$.

{\scriptsize \emph{Hint.} First show that both $-1$ and $-3$ are squares mod $p$.}

\item[G2.] Recall from class that $\pi(x)/x\to 0$ as $x\to\infty$. Using this result, show that for every integer $k>1$, there is a positive integer $n$ with $n/\pi(n)=k$.
\end{enumerate}
 

\end{document}

\documentclass[12pt]{article}
\usepackage{amsmath}
\usepackage{amsfonts}
\usepackage{amssymb}
\usepackage{graphicx}
\usepackage{geometry}
\usepackage{amsthm}
\usepackage{url}
\def\Z{\mathbf{Z}}
\def\Q{\mathbf{Q}}
\def\C{\mathbf{C}}
\def\R{\mathbf{R}}
\def\lcm{\mathrm{lcm}}
\def\ord{\mathrm{ord}}
\def\proj{\mathrm{proj}}
\newcommand{\leg}[2]{\genfrac{(}{)}{}{}{#1}{#2}} 

\renewcommand\subset\subseteq
\newcommand{\vc}[1]{\mathbf{#1}}
\geometry{left=1in, right=1in, top=.75in, bottom=.75in}
\begin{document}
\thispagestyle{empty} \begin{center} {\textbf{MATH 4400/6400 --
Homework \#5}\\ posted \today; due April 8, by midnight}
\end{center}

{\scriptsize \begin{quote} Any fool can know. The point is to understand.\\
    -- Albert Einstein\end{quote}}

 
\vskip 10pt \noindent\textbf{Directions}. Give complete solutions, providing full justifications when appropriate. Your assignment must be stapled if it goes on beyond one page.

\vskip 10pt \noindent\textbf{MATH 4400 problems}
\begin{enumerate}
\item For every positive integer $a$, let $N(a)$ denote the number of integers in the half-open interval $(a,a^2]$ which are divisible by a prime $p > a$.
\begin{enumerate}
\item Prove that $\displaystyle N(a) = \sum_{a < p \le a^2} \lfloor a^2/p\rfloor$.
\item Deduce from (a) and the trivial inequality $N(a) \le a^2$ that $\displaystyle \sum_{a < p \le a^2} \frac{1}{p} \le 2$.
\item By using part (b) several times, find an upper bound on $\displaystyle\sum_{2 < p \le 2^{32}} \frac{1}{p}$.
\end{enumerate}

\item Let $\mathcal{D}(n) = \{d \in \Z^{+}: d \mid n\}$. Let $n_1$ and $n_2$ be relatively prime positive integers. Show that the map
\begin{align*} M\colon \mathcal{D}(n_1) \times \mathcal{D}(n_2) &\to \mathcal{D}(n_1 n_2) \\
    (d_1,d_2) &\mapsto d_1d_2
\end{align*}
is a bijection.

\item Suppose $f$ is a multiplicative function.
\begin{enumerate}
\item Show that $f(1) = 1$ or $f(1)=0$.
\item Show that if $f(1)=0$, then $f(n)=0$ for all positive integers $n$.
\end{enumerate}

\item Show that if $f, g$ are multiplicative functions with $f(1)=g(1)=1$, and $f(p^e) = g(p^e)$ for all primes $p$ and all positive integers $e$, then $f(n)=g(n)$ for all positive integers $n$.

\item Prove that for all positive integers $n$,
\[ \sum_{e \mid n} d(e)^3 = \left(\sum_{e \mid n}d(e)\right)^2.\]
{\scriptsize \emph{Hint.} You may assume the formula $\sum_{k=1}^{m} k^3 = (m(m+1)/2)^2$, which could be proved by induction.}

\item Recall that Euler's $\phi$-function is defined by $$\phi(n) = \#\{m: 1 \le m\le n\text{ and } \gcd(m,n)=1\}.$$
We will show in class that $\phi(n)$ is multiplicative.

Prove: $\phi(n) \sigma(n) \le n^2$ for all $n$.

\item Recall from class that $d_k(n) = \#\{(d_1,\dots,d_k) \in (\Z^{+})^k: d_1\cdots d_k = n\}$. Find a formula for $d_3(n)$ in terms of the prime factorization of $n$.

\item \begin{enumerate}
\item Classify all $n$ for which $\phi(n)$ is an odd number. Justify your answer.
\item Classify all $n$ for which $d(n)$ is an odd number. Justify your answer.
    \item Classify all $n$ for which $\sigma(n)$ is an odd number. Justify your answer.
\end{enumerate}

\end{enumerate}

\vskip 10pt \noindent\textbf{MATH 6400 problems}

\begin{enumerate}
\item[9.] (*) (Euler) Prove that if $n$ is odd and $\sigma(n)$ is twice an odd number, then $n = p^\alpha m^2$ for some prime $p$ and some positive integers $\alpha$ and $m$, where $p\nmid m$. Moreover, $p \equiv \alpha \equiv 1\pmod{4}$.

{\scriptsize \emph{Remark.} The assumptions of the problem hold, in particular, if $n$ is an \emph{odd perfect number}. This was the context in which Euler proved the result.} 


\item[10.] (*) Prove $\phi(n) \sigma(n) \ge \frac{1}{2}n^2$ for all $n$.
\end{enumerate}

\end{document}

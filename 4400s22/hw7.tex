\documentclass[12pt]{article}
\usepackage{amsmath}
\usepackage{amsthm}
\usepackage{amsfonts}
\usepackage{amssymb}
\usepackage{graphicx}
\usepackage{bbm}
\usepackage{geometry}
\def\Z{\mathbb{Z}}
\def\H{\mathbb{H}}
\def\Q{\mathbb{Q}}
\def\R{\mathbb{R}}
\def\Pp{\mathcal{P}}
\def\lcm{\mathrm{lcm}}
\def\proj{\mathrm{proj}}
\def\ord{\mathrm{ord}}
\def\bz{\mathbbm{1}}
\def\1{\mathbbm{1}}
\newcommand{\leg}[2]{\genfrac{(}{)}{}{}{#1}{#2}}
\newcommand{\vc}[1]{\mathbf{#1}}
\theoremstyle{plain}

\newtheorem*{lem}{Lemma}
\newtheorem*{thm}{Theorem}

\theoremstyle{remark}
\newtheorem*{rmk}{Remark}
\geometry{left=1in, right=1in, top=.75in, bottom=.75in}
\makeatletter
\let\@@pmod\pmod
\DeclareRobustCommand{\pmod}{\@ifstar\@pmods\@@pmod}
\def\@pmods#1{\mkern4mu({\operator@font mod}\mkern 6mu#1)}
\makeatother
\begin{document}
\thispagestyle{empty} \begin{center} {\textbf{MATH 4400/6400 --
Homework \#7}\\ posted \today; due May 2, 2022}
\end{center}

\vskip 2pt \noindent\textbf{MATH 4400/6400 problems}.

\begin{enumerate}
\item Show that $\mathcal{I}$ is a subring of $\mathbb{H}$.

{\scriptsize \emph{Hint.} The tricky part is closure under multiplication. For this, let $\delta = \frac{1}{2}(1+i+j+k)$ and argue that $\mathcal{I} = \{A + B i + C j + D \delta: A,B,C,D \in \Z\}$. Then verify that the product of any pair of $1,i,j,\delta$ belongs to $\mathcal{I}$.}

\item \begin{enumerate}
\item Show that the units of $\mathcal{I}$ are precisely the elements of $\mathcal{I}$ of norm $1$.
\item Show that the units in $\mathcal{I}$ are precisely $\pm 1, \pm i, \pm j, \pm k$ and $\frac{\pm 1 \pm i \pm j \pm k}{2}$.
\end{enumerate}

\item Suppose that $\alpha \in \mathcal{I}$. Show that one can find a unit $\epsilon$ of $\mathcal{I}$ for which $\alpha \cdot \epsilon \in \mathcal{L}$.

{\scriptsize \emph{Hint.} If $\alpha \in \mathcal{L}$, take $\epsilon=1$. Otherwise, $\alpha = \frac{1}{2}(a+bi+cj+dk)$ for odd integers $a,b,c,d$. Try $\epsilon = \frac{1}{2}(A-Bi-Cj-Dk)$ where $A, B, C, D$ are all $\pm 1$, with the signs chosen to make $A \equiv a,~B \equiv b,~C \equiv c,~D \equiv d \pmod{4}$.}

\end{enumerate}
\end{document}
